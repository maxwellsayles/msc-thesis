\documentclass{ucalgthes1}   
\usepackage[letterpaper,top=1in, bottom=1.22in, left=1.40in, right=0.850in]{geometry}
\usepackage{fancyhdr}
\fancyhead{}
\fancyfoot{}
\renewcommand{\headrulewidth}{0pt}
\fancyhead[RO,LE]{\thepage}  
\usepackage{hyperref}

\usepackage{algorithm}
\usepackage{algorithmic}
\usepackage{amsfonts}
\usepackage{amssymb}
\usepackage{amsmath}
\usepackage{amsthm}
\usepackage{comment}
\usepackage{float}

\theoremstyle{definition}
\newtheorem{thm}{Theorem}[section]
\newtheorem{lemma}[thm]{Lemma}
\newtheorem{prop}[thm]{Proposition}
\newtheorem{cor}[thm]{Corollary}
\newtheorem{defn}[thm]{Definition}


\renewcommand{\algorithmicrequire}{\textbf{Input:}}
\renewcommand{\algorithmicensure}{\textbf{Output:}}

\newcommand{\CC}{\mathbb{C}}
\newcommand{\NN}{\mathbb{N}}
\newcommand{\RR}{\mathbb{R}}
\newcommand{\KK}{\mathbb{K}}
\newcommand{\MM}{\mathcal{M}}
\newcommand{\OO}{\mathcal{O}}
\newcommand{\ZZ}{\mathbb{Z}}
\newcommand{\QQ}{\mathbb{Q}}
\newcommand{\NP}{\textrm{NP}}
\newcommand{\RRgtz}{\mathbb{R}_{>0}}
\newcommand{\ZZgtz}{\mathbb{Z}_{>0}}
\newcommand{\ZZgez}{\mathbb{Z}_{\ge 0}}
\newcommand{\QQgtz}{\mathbb{Q}_{>0}}
\newcommand{\QQgez}{\mathbb{Q}_{\ge 0}}
\newcommand{\matrixto}[2]{\left[ \begin{array}{rr} #1 & #2 \end{array} \right]}
\newcommand{\matrixot}[2]{\left[ \begin{array}{r} #1 \\ #2 \end{array} \right]}
\newcommand{\matrixtt}[4]{\left[ \begin{array}{rr} #1 & #2 \\ #3 & #4 \end{array} \right]}
\newcommand{\ntoinfty}{\lim_{n \rightarrow \infty}}
\newcommand{\floor}[1]{\left\lfloor #1 \right\rfloor}
\newcommand{\ceil}[1]{\left\lceil #1 \right\rceil}

%\setlength{\parindent}{0pt}
%\setlength{\parskip}{2ex} 

\begin{document}

\setcounter{chapter}{1}
\chapter{Ideal Arithmetic}
\label{chap:idealArithmetic}

The focus of this thesis is fast arithmetic and fast exponentiation in the ideal class group of imaginary quadratic number fields.  We begin with the relevant theory of quadratic number fields, then discuss quadratic orders and ideals of quadratic orders.  Finally, we discuss arithmetic in the ideal class group.  The theory presented here is available in detail in reference texts on algebraic number theory such as \cite{IR90}, \cite{Hua82}, or \cite{Coh80}. 



%%%%%%%%%%%%%%%%%%%%%
% QUADRATIC NUMBERS %
%%%%%%%%%%%%%%%%%%%%%
\section{Quadratic Numbers}

The ideal class group of imaginary quadratic fields begins with quadratic numbers, a type of algebraic number, so we begin there.   

\begin{defn}
An \emph{algebraic number} is a root $\alpha$ of a polynomial $f(x) \in \QQ[x]$ with rational coefficients.
\end{defn}

\noindent
Given an algebraic number $\alpha$, we create an algebraic number field $\KK$ as the extension field of $\QQ$ obtained by adjoining $\alpha$, denoted $\QQ(\alpha)$. According to \cite[p.76]{Jacobson2009}, we can write $\KK$ as 
\[
	\KK = \QQ(\alpha) = \left\{ \frac{f(\alpha)}{g(\alpha)} : f(x), g(x) \in \QQ[x]; g(\alpha) \ne 0 \right\}.
\]

\noindent
A rational number $a/b$ for integers $a$ and $b$ is the root of a polynomial $bx-a$ where the highest degree of its terms is 1.  As such, rational numbers are degree 1 algebraic numbers. Given a polynomial $f(x) = ax^2 + bx + c$ with integer coefficients, the highest degree of its terms is 2.  A root $\alpha$ of $f(x)$ above is a degree 2 algebraic number and is called a \emph{quadratic number}.  The roots are given by the quadratic formula
\[
	\alpha = \frac{-b \pm \sqrt{b^2 - 4ac}}{2a} ~.
\]
The \emph{discriminant} of $f(x)$ is $\Delta = b^2 - 4ac$, and our quadratic number field $\KK$ is 
\[
	\KK = \QQ(\alpha) = \QQ(\sqrt{\Delta}) = \{u + v\sqrt{\Delta} : u,v \in \QQ\}.
\]
When $\Delta$ is positive, $\KK$ is a subset of the real numbers, $\RR$, and is a \emph{real} quadratic number field. When $\Delta$ is negative, $\KK$ is a subset of the complex numbers, $\CC$, and is an \emph{imaginary} quadratic number field.  In this thesis we concern ourselves only with the imaginary case.  Notice that if $\Delta = f^2 \Delta_0$ for some $f \in \ZZgtz$ where $\Delta_0$ is square-free, then $\QQ(\sqrt{\Delta}) = \QQ(\sqrt{\Delta_0})$. For $\Delta_0$ to be square-free, it must be that $\Delta_0 \not\equiv 0 \pmod 4$ since this would imply that $\Delta_0$ is divisible by 4, a perfect square.  


%%%%%%%%%%%%%%%%%%%%%%
% QUADRATIC INTEGERS %
%%%%%%%%%%%%%%%%%%%%%%
\bigbreak
\section{Quadratic Integers}

A polynomial with a leading coefficient of 1 is a \emph{monic} polynomial. The root $\alpha$ of a monic polynomial $f(x)$ with integer coefficients is an \emph{algebraic integer}, and when $f(x)$ is a monic quadratic polynomial with integer coefficients, the root $\alpha$ is a \emph{quadratic integer}. By \cite[p.77]{Jacobson2009} we have the following theorem:

\begin{thm}
A quadratic integer $\alpha$ is an algebraic integer of $\QQ(\sqrt{\Delta_0})$ where $\alpha$ can be written as $\alpha = x + y \omega_0$ for $x, y \in \ZZ$ where
\begin{equation*}
	\omega_0 = \begin{cases}
		\sqrt{\Delta_0} & \textrm{ when } \Delta_0 \not\equiv 1 \pmod 4 \\
		\frac{1+\sqrt{\Delta_0}}{2} & \textrm{ when } \Delta_0 \equiv 1 \pmod 4.
	\end{cases}
\end{equation*}
\end{thm}


%%%%%%%%%%%%%%%%%
% MAXIMAL ORDER %
%%%%%%%%%%%%%%%%%
\bigbreak
\section{Maximal Order of Algebraic Integers}

A $\ZZ$-module, $\MM$, is a an additive abelian subgroup of an additive abelian group, $A$.

\begin{defn}
When $X = \{ \xi_1, \xi_2, \xi_3, ..., \xi_n \}$ is a subset of a number field $\KK$, then the set generated by all linear combinations of the elements of $X$ with all possible integer coefficients is the finitely generated submodule of $X$:
\[
	\MM = \left \{ \sum_{i}^n x_i \xi_i : x_i \in \ZZ, \xi_i \in X \right \},
\]
and is denoted as
\[
	\MM = [ \xi_1, \xi_2, ..., \xi_n ]
\]
or equivalently
\[
	\MM =  \xi_1 \ZZ + \xi_2 \ZZ + \cdots + \xi_n \ZZ.
\]
\end{defn}

\begin{thm}
An \emph{quadratic order} $\OO_\Delta$ of $\QQ(\Delta)$ is a $\ZZ$-module of $\QQ(\Delta)$ such that $\OO_\Delta$ is a subring of $\QQ(\Delta)$ containing 1.  According to \cite[p.81]{Jacobson2009}, we can write $\OO_\Delta$ as
\[
	\left[ 1, \frac{\Delta + \sqrt{\Delta}}{2} \right] = [1, f\omega_0].
\]
When $\OO_\Delta = [1, \omega_0]$ the order is \emph{maximal} since any order $\OO = [1, f\omega_0]$ is a subring of $\OO_\Delta$. 
\end{thm}


%%%%%%%%%%
% IDEALS %
%%%%%%%%%%
\bigbreak
\section{Ideals of $\OO_\Delta$}

\begin{defn}
Let $\mathfrak a$ be an additive subgroup of the order $\OO$.  For any $\xi \in \OO$, define $\xi \mathfrak a$ to be the set $\{\xi a : a \in \mathfrak a\}$.  When $\xi \mathfrak a \subseteq \mathfrak a$, then $\mathfrak a$ is an \emph{ideal} of the order $\OO$, also called an \emph{$\OO$-ideal}.
\end{defn}

\begin{thm}
\label{thm:idealZModule}
When $\OO_\Delta = [1, f\omega_0]$ is the order of a quadratic field, a non-zero ideal $\mathfrak a$ of $\OO_\Delta$ can be uniquely written as a two dimensional $\ZZ$-module 
\[
	\mathfrak a = s\left[a, \frac{b+\sqrt{\Delta}}{2} \right]
\]
for $s, a, b \in \ZZ, s > 0, a > 0$, and $b$ is unique $\bmod ~2a$ \cite[p.86]{Jacobson2009}. When $s = 1$ the ideal $\mathfrak a$ is \emph{primitive}.
\end{thm}

Prime ideals allow us to easily create an ideal.  For a prime ideal $\mathfrak p \in \OO_\Delta$ it can be shown (\cite[p.19]{Jacobson1999}) that $\mathfrak p \cap \ZZ = p\ZZ$ for some prime integer $p \in \ZZ$. Let
\[
	\mathfrak p = s\left[a, \frac{b + \sqrt{\Delta}}{2}\right]
\]
and it follows that either $s=p$ and $a=1$, or $s=1$ and $a=p$.  In the first case $\mathfrak p = p\OO_\Delta$, while in the second case $b = \pm \sqrt{\Delta} \pmod p$ and $\mathfrak p = [p, (b + \sqrt{\delta})/2]$ when $4p ~|~ b^2 - \Delta$.  Algorithm \ref{alg:prime} gives pseudo-code.

% PRIME 
\begin{algorithm}[h]
\caption{Prime Ideal}
\label{alg:prime}
\begin{algorithmic}[1]
\REQUIRE A prime integer $p \in \ZZ$.
\ENSURE A representative $\mathfrak p = [p, (b+\sqrt\Delta)/2]$ such that $\mathfrak p$ is a prime ideal if one exists.
\STATE $b \gets \textrm{the positive } \sqrt\Delta \pmod p$
\IF {$4p ~|~ b^2-\Delta$}
	\RETURN $[p, (b+\sqrt\Delta)/2]$
\ENDIF
\STATE $b \gets p-b$
\IF {$4p ~|~ b^2-\Delta$}
	\RETURN $[p, (b+\sqrt\Delta)/2]$
\ENDIF
\RETURN none
\end{algorithmic}
\end{algorithm}

The following details of ideals are summarized from \cite[pp.14,15]{Jacobson1999} and \cite[p.88]{Jacobson2009}. An ideal $\mathfrak{a}$ generated by a single element $\alpha \in \OO$ is \emph{principal}, and is written $\mathfrak{a} = \alpha \OO_\Delta = (\alpha)$.  Two ideals $\mathfrak a$ and $\mathfrak b$ are equivalent if there exists $\alpha, \beta \in \OO_\Delta$ such that $\alpha \beta \neq 0$ and $(\alpha)\mathfrak a = (\beta) \mathfrak b$ \cite[p.88]{Jacobson2009}. We denote by $[\mathfrak a]$ the \emph{ideal class} of all ideals equivalent to $\mathfrak a$. The \emph{identity} ideal is $\OO_\Delta = [1, f\omega_0]$ since $\mathfrak a = \mathfrak a \OO_\Delta = \OO_\Delta \mathfrak a$.  An ideal $\mathfrak a = s[a, (b+\sqrt{\Delta})/2]$ is \emph{invertible} if there exists an ideal $\mathfrak b$ such that $\mathfrak a \mathfrak b = [\OO_\Delta]$, and an ideal $\mathfrak b$ exists when $\gcd(a, b, (b^2-\Delta)/(4a)) = 1$ \cite[p.14]{Jacobson1999}. The inverse is given by \cite[pp.14,15]{Jacobson1999}
\[
	{\mathfrak a}^{-1} = \frac{s}{\mathcal N(\mathfrak a)} \left[a, \frac{-b+\sqrt{\Delta}}{2} \right]
\]
where $\mathcal N(\mathfrak a) = s^2a$ is the norm of $\mathfrak a$ and is multiplicative. When $\OO_\Delta$ is maximal, all ideals of $\OO_\Delta$ are invertible.


%%%%%%%%%%%%%%%%%%%%%
% IDEAL CLASS GROUP %
%%%%%%%%%%%%%%%%%%%%%
\bigbreak
\section{Ideal Class Group}

The \emph{ideal class group}, $Cl_\Delta$, is the set of all equivalence classes of invertible $\OO$-ideals.  The group operation is defined as the product of two class representatives.  In \ref{subsec:reduction}, we state what it is for an ideal to be in reduced form, and in subsection \ref{subsec:idealMultiply}, we show how to multiply two reduced class representatives, and then in \ref{subsec:nucomp} we discuss how to perform multiplication such that the result is a reduced or almost reduced representative.  We then extend this to the case of computing the square (\ref{subsec:nudupl}) and cube (\ref{subsec:nucube}) of an ideal class.  Finally, in section \ref{section:computerRepresentation} (TODO) we give a concrete representation of the ideal class group of imaginary quadratic fields and the operations on the elements of this group. 

%%%%%%%%%%%%%
% REDUCTION %
%%%%%%%%%%%%%
\subsection{Reduced Representatives}
\label{subsec:reduction}
\begin{defn}
Let $\mathfrak{a} = [a, (b+\sqrt{\Delta})/2]$ be a primitive $\OO$-ideal. Then $\mathfrak{a}$ is \emph{reduced} when $-a < b \le a$ \cite[p.13]{Jacobson1999}.
\end{defn}

Suppose $\mathfrak a = [a, (b + \sqrt\Delta)/2]$. The following will compute a reduced representative of the ideal class $[\mathfrak a]$.  Let $c = (b^2 - \Delta)/4a$.  If $a > c$ compute ${\mathfrak a}^{-1}$ by swapping $a$ and $c$ and letting $b = -b$.  Since $b$ is unique $\bmod{~2a}$, we reduce $b \bmod{2a}$ and repeat while $\mathfrak a$ is not reduced.  Pseudo-code is given in Algorithm \ref{alg:reduce}. (TODO: Justify that this is correct, and cite the routine).

% REDUCE
\begin{algorithm}[h]
\caption{Reduce}
\label{alg:reduce}
\begin{algorithmic}[1]
\REQUIRE An ideal class representative $\mathfrak a_1 = [a_1, (b_1+\sqrt\Delta)/2]$ and $c_1 = ({b_1}^2 - \Delta)/4a_1$.
\ENSURE A reduced representative $\mathfrak a = [a, (b+\sqrt\Delta)/2]$.
\STATE $(a, b, c) \gets (a_1, b_1, c_1)$
\WHILE {$a > c$ or $b > a$ or $b \le -a$}
	\IF {$a > c$}
		\STATE swap $a$ with $c$ and let $b \gets -b$
	\ENDIF
	\IF {$b > a$ or $b \le -a$}
		\STATE $b \gets b'$ such that $-a < b' \le a$ and $b' \equiv b \pmod{2a}$
		\STATE $c \gets (b^2-\Delta)/4a$
	\ENDIF
\ENDWHILE
\IF {$a=c$ and $b < 0$}
	\STATE $b \gets -b$
\ENDIF
\RETURN $[a, (b+\sqrt\Delta)/2]$
\end{algorithmic}
\end{algorithm}



%%%%%%%%%%%%%%%%%%
% MULTIPLICATION %
%%%%%%%%%%%%%%%%%%
\subsection{Multiplication of Ideal Classes}
\label{subsec:idealMultiply}

The ideal class group operation is multiplication of ideal class representatives. Given two representative ideals $\mathfrak a = [a_1, (b_1 + \sqrt{\Delta})/2]$ and $\mathfrak b = [a_2, (b_2 + \sqrt{\Delta})/2]$ in reduced form, the (non-reduced) product $\mathfrak a \mathfrak b$ can computed using:
\begin{eqnarray}
	c_2 & = & ({b_2}^2-\Delta)/4a_2 \\
	s & = & Ya_1 + Va_2 + W(b_1+b_2)/2 \textrm{ for $Y, V, W \in \ZZ$}  \label{eq:idealProductS} \\
	U & = & V(b_1-b_2)/2 - Wc_2                                         \label{eq:idealProductU} \\
	a & = & (a_1a_2)/s^2                                                \label{eq:idealProductA} \\
	b & = & (b_2 + 2Ua_2/s) \bmod{2a}                                   \label{eq:idealProductB} \\
	\mathfrak a \mathfrak b & = & s\left[a, \frac{b + \sqrt{\Delta}}{2}\right].
\end{eqnarray}

The remainder of this subsection is used to derive the above equations and may be skipped by the uninterested reader.  We adapt much of the presentation given in \cite[pp.117,118]{Jacobson2009}. Component-wise multiplication of $\mathfrak a$ and $\mathfrak b$ give us
\begin{equation}
\begin{split}
	\mathfrak{a} \mathfrak{b} & = a_1a_2 \ZZ + a_1 \frac{b_2 + \sqrt{\Delta}}{2} \ZZ + a_2 \frac{b_1 + \sqrt{\Delta}}{2} \ZZ + \frac{b_1 + \sqrt{\Delta}}{2} \cdot \frac{b_2 + \sqrt{\Delta}}{2} \ZZ \\
	& = a_1a_2 \ZZ + \frac{a_1b_2 + a_1\sqrt{\Delta}}{2} \ZZ + \frac{a_2b_1 + a_2\sqrt{\Delta}}{2} \ZZ + \frac{b_1b_2 + (b_1+b_2)\sqrt{\Delta} + \Delta}{4} \ZZ \label{eq:productExpanded}
\end{split}
\end{equation}

\noindent
The goal is to reformulate this given the representation in Theorem \eqref{thm:idealZModule} as
\[
	\mathfrak{a} \mathfrak{b} = sa \ZZ + s \left(\frac{b + \sqrt{\Delta}}{2}\right) \ZZ
\]
for some $s, a, b \in \ZZ$.  By the multiplicative property of the norm we have
\begin{eqnarray*}
	&& s^2a = N(\mathfrak{a}\mathfrak{b}) = N(\mathfrak{a})N(\mathfrak{b}) = a_1 a_2 \\
	& \Rightarrow & a = \frac{a_1a_2}{s^2},
\end{eqnarray*}
which gives us Equation \ref{eq:idealProductA}. Now, by the second term of equation \eqref{eq:productExpanded} we know that $(a_1b_2 + a_1\sqrt{\Delta})/2 \in \mathfrak{a}\mathfrak{b}$.  It follows that there is some $x,y \in \ZZ$ such that
\[
	\frac{a_1b_2 + a_1\sqrt{\Delta}}{2} = xsa + ys\left(\frac{b+\sqrt{\Delta}}{2}\right).
\]
Equating irrational parts we have
\begin{equation*}
	\frac{a_1\sqrt{\Delta}}{2} = \frac{ys\sqrt{\Delta}}{2}
\end{equation*}
\noindent
Hence, $s ~|~ a_1$.  Similarly, by the third and fourth terms of equation \eqref{eq:productExpanded} we have $(a_2b_1+a_2\sqrt{\Delta})/2 \in \mathfrak{a}\mathfrak{b}$ which implies that $s~|~a_2$ and $(b_1b_2 + (b_1+b_2)\sqrt{\Delta} + \Delta)/4 \in \mathfrak{a}\mathfrak{b}$ which implies that $s~|~(b_1+b_2)/2$. 

By the second generator $s(b+\sqrt\Delta)/2$ of $\mathfrak{a}\mathfrak{b}$ and the entire right hand side of equation \eqref{eq:productExpanded} there exists $X, Y, V, W \in \ZZ$ such that
\begin{equation}
\label{eq:productSecond}
\begin{split}
	\frac{sb+s\sqrt\Delta}{2} & = Xa_1a_2 + Y\frac{a_1b_2+a_1\sqrt\Delta}{2} + V\frac{a_2b_1 + a_2\sqrt{\Delta}}{2} + W\frac{b_1b_2 + (b_1+b_2)\sqrt{\Delta} + \Delta}{4} \\
	& = Xa_1a_2 + Y\frac{a_1b_2}{2} + V\frac{a_2b_1}{2} + W\frac{b_1b_2 + \Delta}{4} + \left(Y\frac{a_1}{2} + V\frac{a_2}{2} + W\frac{b_1+b_2}{4}\right)\sqrt\Delta. 
\end{split}
\end{equation}

\noindent
Again, by equating irrational parts we have
\begin{align}
	\frac{s\sqrt\Delta}{2} & = \left(Y\frac{a_1}{2} + V\frac{a_2}{2} + W\frac{b_1+b_2}{4}\right)\sqrt\Delta \nonumber \\
	s & = Ya_1 + Va_2 + W\frac{b_1+b_2}{2}. \label{eq:sAsGCD}
\end{align}
Since $s~|~\gcd(a_1, a_2, (b_1+b_2)/2)$, we have that $s = \gcd(a_1, a_2, (b_1+b_2)/2)$, which is Equation \ref{eq:idealProductS}.  

It remains to compute $b \pmod{2a}$.  Recall that $a = a_1a_2/s^2$.  This time, by equating the rational parts of \eqref{eq:productSecond} we have:
\begin{align}
	\frac{sb}{2} & = Xa_1a_2 + Y\frac{a_1b_2}{2} + V\frac{a_2b_1}{2} + W\frac{b_1b_2 + \Delta}{4} \nonumber \\
	b & = 2X\frac{a_1a_2}{s} + Y\frac{a_1b_2}{s} + V\frac{a_2b_1}{s} + W\frac{b_1b_2 + \Delta}{2s} \nonumber \\
	b & \equiv Y\frac{a_1b_2}{s} + V\frac{a_2b_1}{s} + W\frac{b_1b_2 + \Delta}{2s} \pmod{2a} \label{eq:bMod2a}
\end{align}

\noindent
This gives us $b$.  However, we can rewrite \eqref{eq:bMod2a} with fewer multiplies and divides.  By equation \eqref{eq:sAsGCD}, we have
\begin{align*}
	s & = Ya_1 + Va_2 + W\frac{b_1+b_2}{2} \\
	1 & = Y\frac{a_1}{s} + V\frac{a_2}{s} + W\frac{b_1+b_2}{2s} \\
	Y\frac{a_1}{s} & = 1 - V\frac{a_2}{s} - W\frac{b_1+b_2}{2s}.
\end{align*}

\noindent
Substituting into equation \eqref{eq:bMod2a} we get
\begin{alignat*}{2}
	b & \equiv b_2(1-V\frac{a_2}{s} - W\frac{b_1+b_2}{2s}) + V\frac{a_2b_1}{s} + W\frac{b_1b_2 + \Delta}{2s} && \pmod{2a} \\
	& \equiv b_2 - V\frac{a_2b_2}{s} - W\frac{b_1b_2+{b_2}^2}{2s} + V\frac{a_2b_1}{s} + W\frac{b_1b_2 + \Delta}{2s} && \pmod{2a} \\
	& \equiv b_2 + V\frac{a_2(b_1-b_2)}{s} + W\frac{\Delta - {b_2}^2}{2s} && \pmod{2a} \\
	& \equiv b_2 + V\frac{2a_2(b_1-b_2)}{2s} + W\frac{2a_2(\Delta - {b_2}^2)}{2a_2 \cdot 2s} && \pmod{2a} \\
	& \equiv b_2 + \frac{2a_2}{s} \left( V\frac{b_1-b_2}{2} + W\frac{\Delta - {b_2}^2}{4a} \right) && \pmod{2a}.
\end{alignat*}
Let $c_2 = ({b_2}^2 - \Delta)/4a_2$ and $U = V(b_1-b_2)/2 - Wc_2$ and we have
\[
	b \equiv b_2 + \frac{2a_2}{s} U \pmod{2a},
\]
which completes the derivation of Equation \ref{eq:idealProductB}.


%%%%%%%%%%
% NUCOMP %
%%%%%%%%%%
\subsection{Fast Ideal Multiplication (NUCOMP)}
\label{subsec:nucomp}

Shanks gives an algorithm for multiplying two ideal class representatives such that their product is reduced or almost reduced.  The algorithm is known as NUCOMP and stands for ``New COMPosition''.  This algorithm is often faster in practice as the intermediate numbers are smaller and the final product requires fewer (often no) applications of the reduction operator to be converted to reduced form. The description of NUCOMP provided here is a high level description of the algorithm based on \cite[pp.119-123]{Jacobson09}.

Rather than computing the product form directly and then computing the reduced ideal representative, observe that
\[
	\frac{b}{a} = \frac{b_2 + Ua_2/s}{(a_1/s)(a_2/s)} 
	= \frac{s^2 b_2+sUa_2}{a_1a_2}
	\approx \frac{sU}{a_1}
\]
where $U$ is given by Equation \ref{eq:idealProductU}.  Compute the continued fraction expansion of $sU/a_1 = \langle q_0, q_1, \dots, q_i, \phi_{i+1} \rangle$ using
\begin{align*}
	q_i &= \floor{R_{i-2} / R_{i-1}}, &
	R_i &= R_{i-2} - q_i R_{i-1}, &
	C_i &= C_{i-2} - q_i C_{i-1}
\end{align*}
where
\[
	\matrixtt{R_{-2}}{R_{-1}}{C_{-2}}{C_{-1}} = \matrixtt{sU}{a_1}{-1}{0}
\]
and by \cite[Theorem 5.43]{Jacobson09}, select $i$ such that
\[
	R_i < \sqrt{a_1/a_2} ~ |\Delta/4|^{1/4} < R_{i-1}.
\]
We then compute
\begin{align*}
	M_1 &= \frac{R_i a_2 + sC_i(b_1-b_2)/2}{a_1}  \in \ZZ, &
	M_2 &= \frac{R_i (b_1+b_2)/2 - s C_i c_2}{a_1} \in \ZZ, \\
	a &= (-1)^{i+1} (R_i M_1  - C_i M_2), &
	b &\equiv \frac{R_i a_2 /s + C_{i-1} a}{C_i} \pmod{2a}
\end{align*}
where $c_2 = ({b_2}^2-\Delta)/4a_2$ \cite[Equation 5.44]{Jacobson09}.  We give pseudocode for the complete multiplication in Algorithm \ref{alg:nucomp}.

% NUCOMP
\begin{algorithm}[h]
\caption{NUCOMP. Based on \cite[pp.441-443]{Jacobson2009}.}
\label{alg:nucomp}
\begin{algorithmic}[1]
\REQUIRE Reduced representatives $\mathfrak a = [a_1, (b_1+\sqrt\Delta)/2]$, $\mathfrak b = [a_2, (b_2+\sqrt\Delta)/2]$ with \break $c_1 = ({b_1}^2-\Delta)/4a_1$, $c_2 = ({b_2}^2-\Delta)/4a_2$, and discriminant $\Delta$.
\ENSURE A reduced or almost reduced representative $\mathfrak a \mathfrak b$.
\STATE $s \gets Ya_1 + Va_2 + W (b_1+b_2)/2$ for $Y, V, W \in \ZZ$
\STATE $U \gets V(b_1-b_2)/2 + Wc_2$
\STATE $\matrixtt{R_{-2}}{R_{-1}}{C_{-2}}{C_{-1}} \gets \matrixtt{sU}{a_1}{-1}{0}$
\STATE Find $i$ such that $R_i < \sqrt{a_1/a_2} ~ |\Delta/4|^{1/4} < R_{i-1}$ using the recurrences: \\
$q_i = \floor{R_{i-2}/R_{i-1}}$ \\
$R_i = R_{i-2}-q_i R_{i-1}$ \\
$C_i=C_{i-2}-q_i C_{i-1}$
\STATE $M_1 \gets (R_i a_2 + sC_i(b_1-b_2)/2)/a_1$
\STATE $M_2 \gets (R_i (b_1+b_2)/2 -sC_i c_2)/a_1$
\STATE $a \gets (-1)^{i+1}(R_i M_1 - C_i M_2)$
\STATE $b \gets ((R_i a_2/s + C_{i-1} a)/C_i) \bmod{2a}$
\RETURN $[a, (b+\sqrt\Delta)/2]$
\end{algorithmic}
\end{algorithm}


%%%%%%%%%%
% NUDUPL %
%%%%%%%%%%
\subsection{Fast Ideal Squaring (NUDUPL)}\label{subsec:nudupl}

In the case of ideal class squaring, much of the arithmetic used to multiply two ideal classes can be simplified since in this case $a_1=a_2$ and $b_1=b_2$.  Let 
\[
	s = Xa_1 + Yb_1
\]
for $X,Y \in \ZZ$ using the extended Euclidean algorithm; and let 
\[
	U = Yc_1.
\]
As before, we compute the continued fraction expansion of $sU/a_1$, but the bound on $R_i$ simplifies to
\[
	R_i < |\Delta/4|^{1/4} < R_{i-1}.
\]
Computing the ideal class representative simplifies as well.  We have
\begin{align*}
	M_1 &= R_i, & 
	M_2 &= \frac{R_i b_1 - sC_i c_1}{a_1}, \\
	a &= (-1)^{i+1}({R_i}^2 - C_i M_2), &
	b &\equiv \frac{R_i a_1/s  + C_{i-1} a}{C_i} \pmod{2a}.
\end{align*}
As before, the representative $[a, (b+\sqrt\Delta)/2]$ is either reduced or almost reduced.  \break Pseudocode for our implementation is given in algorithm \ref{alg:nudupl}.


%%%%%%%%%%
% NUCUBE %
%%%%%%%%%%
\subsection{Fast Ideal Cubing (NUCUBE)}\label{subsec:nucube}

Common exponentiation methods using binary representations of exponents make use of squaring and multiplication operations.  If we consider binary-ternary representations of exponents, cubing becomes practical.  In general, if we want to compute ${\mathfrak a}^3$ for an ideal class representative $\mathfrak a = [a, (b+\sqrt\Delta)/2]$, we can take advantage of the simplification that happens when expanding the computation of ${\mathfrak a}^2 \mathfrak a$.  Here we provide a high level description of a technique for cubing based on similar ideas to NUCOMP and NUDUPL, namely that of computing the quotients of a continued fraction expansion useful to computing a reduced ideal representative.  A detailed description and analysis of this technique can be found in \cite{Ijs2010}.

An algorithm for ideal cubing in quadratic number fields is given as algorithm \ref{alg:idealCube} and is modified from \cite[Appendix A, Algorithm 5]{Ijs2010} so as to conform to the conventions used in this thesis. To modify this algorithm to compute an almost reduced ideal, as in NUCOMP, we check before step 13 if $L < \sqrt{|a_1|} ~ |\Delta/4|^{1/4}$.  If so, we complete the cube as in algorithm \ref{alg:idealCube}.  Otherwise, as in NUCOMP and NUCUBE, we compute the continued fraction expansion of $L/K = \langle q_0, q_1, \dots, q_i, \phi_{i+1}\rangle$ for $i$ such that
\[
	R_i < \sqrt{|a_1|} ~ |\Delta/4|^{1/4} < R_{i-1}.
\]

\noindent
Having computed $R_i$ and $C_i$, the reduced or almost reduced ideal ${\mathfrak a}^3$ can be computed as follows:
\begin{align*}
	b_2 &= b_1 -2NK \pmod L, & \\
	M_1 &= (NR_i + (b_2-b_1)C_i)/L, &
	M_2 &= (R_i(b_1+b_2)+c_1s_2)/L, \\
	a &= (-1)^{i-1}(R_iM_1-C_iM_2), &
	b &= (NR_i + aC_{i-1})/C_i-b_1.
\end{align*}
Then, $\mathfrak a^3 = [a, (b+\sqrt\Delta)/2]$.  Pseudo-code for the complete NUCUBE used is given in Algorithm \ref{alg:nucube}.




\end{document}
