\documentclass{ucalgthes1}   
\usepackage[letterpaper,top=1in, bottom=1.22in, left=1.40in, right=0.850in]{geometry}
\usepackage{fancyhdr}
\fancyhead{}
\fancyfoot{}
\renewcommand{\headrulewidth}{0pt}
\fancyhead[RO,LE]{\thepage}  
\usepackage{hyperref}

\usepackage{algorithm}
\usepackage{algorithmic}
\usepackage{amsfonts}
\usepackage{amssymb}
\usepackage{amsmath}
\usepackage{amsthm}
\usepackage{comment}
\usepackage{float}

\theoremstyle{definition}
\newtheorem{thm}{Theorem}[section]
\newtheorem{lemma}[thm]{Lemma}
\newtheorem{prop}[thm]{Proposition}
\newtheorem{cor}[thm]{Corollary}
\newtheorem{defn}[thm]{Definition}


\renewcommand{\algorithmicrequire}{\textbf{Input:}}
\renewcommand{\algorithmicensure}{\textbf{Output:}}

\newcommand{\CC}{\mathbb{C}}
\newcommand{\NN}{\mathbb{N}}
\newcommand{\RR}{\mathbb{R}}
\newcommand{\KK}{\mathbb{K}}
\newcommand{\MM}{\mathcal{M}}
\newcommand{\OO}{\mathcal{O}}
\newcommand{\ZZ}{\mathbb{Z}}
\newcommand{\QQ}{\mathbb{Q}}
\newcommand{\NP}{\textrm{NP}}
\newcommand{\RRgtz}{\mathbb{R}_{>0}}
\newcommand{\ZZgtz}{\mathbb{Z}_{>0}}
\newcommand{\ZZgez}{\mathbb{Z}_{\ge 0}}
\newcommand{\QQgtz}{\mathbb{Q}_{>0}}
\newcommand{\QQgez}{\mathbb{Q}_{\ge 0}}
\newcommand{\matrixto}[2]{\left[ \begin{array}{rr} #1 & #2 \end{array} \right]}
\newcommand{\matrixot}[2]{\left[ \begin{array}{r} #1 \\ #2 \end{array} \right]}
\newcommand{\matrixtt}[4]{\left[ \begin{array}{rr} #1 & #2 \\ #3 & #4 \end{array} \right]}
\newcommand{\ntoinfty}{\lim_{n \rightarrow \infty}}
\newcommand{\floor}[1]{\left\lfloor #1 \right\rfloor}
\newcommand{\ceil}[1]{\left\lceil #1 \right\rceil}

%\setlength{\parindent}{0pt}
%\setlength{\parskip}{2ex} 

\begin{document}

\setcounter{chapter}{1}
\chapter{Ideal Arithmetic}
\label{chap:idealArithmetic}

A focus of this thesis is arithmetic and exponentiation in the ideal class group of imaginary quadratic number fields.  We begin with the relevant theory of quadratic number fields, then discuss quadratic orders and ideals of quadratic orders.  Finally, we discuss arithmetic in the ideal class group.  The theory presented here is available in detail in reference texts on algebraic number theory such as \cite{Ireland1990}, \cite{Hua82}, or \cite{Cohn1980}. 



%%%%%%%%%%%%%%%%%%%%%
% QUADRATIC NUMBERS %
%%%%%%%%%%%%%%%%%%%%%
\section{Quadratic Numbers}

A \emph{quadratic number} is a root $\alpha$ of a quadratic polynomial $f(x) = ax^2 + bx + c$ with integer coefficients. The roots of $f(x)$ are given by the quadratic formula
\[
	\alpha = \frac{-b \pm \sqrt{b^2 - 4ac}}{2a} ~.
\]
The \emph{discriminant} of $f(x)$ is $\Delta = b^2 - 4ac$, and we can construct a quadratic number field $\KK$ as an extension field of the rational numbers $\QQ$ as
\[
	\KK = \QQ(\alpha) = \QQ(\sqrt{\Delta}) = \{u + v\sqrt{\Delta} : u,v \in \QQ\}.
\]
When $\Delta$ is positive, $\KK$ is a subset of the real numbers, $\RR$, and is a \emph{real} quadratic number field. When $\Delta$ is negative, $\KK$ is a subset of the complex numbers, $\CC$, and is an \emph{imaginary} quadratic number field.  In this thesis we concern ourselves only with the imaginary case.  Notice that if $\Delta = f^2 \Delta_0$ for some $f \in \ZZgtz$ where $\Delta_0$ is square-free, then $\QQ(\sqrt{\Delta}) = \QQ(\sqrt{\Delta_0})$. For $\Delta_0$ to be square-free, it must be that $\Delta_0 \not\equiv 0 \pmod 4$ since this would imply that $\Delta_0$ is divisible by 4, a perfect square.  


%%%%%%%%%%%%%%%%%%%%%%
% QUADRATIC INTEGERS %
%%%%%%%%%%%%%%%%%%%%%%
\bigbreak
\section{Quadratic Integers}

A polynomial with a leading coefficient of 1 is a \emph{monic} polynomial, and the root $\alpha$ of a monic polynomial $f(x)$ with integer coefficients is an \emph{algebraic integer}. The rational numbers are degree 1 algebraic numbers since they are roots of degree 1 polynomials $bx-a$.  The roots of monic degree 1 polynomials $x-a$ are the integers, $\ZZ$.  When $f(x)$ is a monic quadratic polynomial with integer coefficients, the root $\alpha$ is a \emph{quadratic integer}. By \cite[p.77]{Jacobson2009} we have the following theorem:

\begin{thm}
A quadratic integer $\alpha$ is an algebraic integer of $\QQ(\sqrt{\Delta_0})$ where $\alpha$ can be written as $\alpha = x + y \omega_0$ for $x, y \in \ZZ$ where
\begin{equation*}
	\omega_0 = \begin{cases}
		\sqrt{\Delta_0} & \textrm{ when } \Delta_0 \equiv 2, 3 \pmod 4 \\
		\frac{1+\sqrt{\Delta_0}}{2} & \textrm{ when } \Delta_0 \equiv 1 \pmod 4.
	\end{cases}
\end{equation*}
\end{thm}


%%%%%%%%%%%%%%%%%
% MAXIMAL ORDER %
%%%%%%%%%%%%%%%%%
\bigbreak
\section{Maximal Order of Algebraic Integers}

The \emph{maximal order} of a field $\KK$ is the set of all algebraic integers contained within $\KK$.  We use the notation of $\ZZ$-modules to characterize the maximal order of quadratic integers.


\begin{defn}
Let $X = \{ \xi_1, \xi_2, \xi_3, ..., \xi_n \}$ be a subset of a number field $\KK$.  A \emph{$\ZZ$\mbox{-}module}, $\MM$, is a finitely generated additive abelian group such that
\begin{align*}
	\MM &= [ \xi_1, \xi_2, ..., \xi_n ] \\
	& =  \xi_1 \ZZ + \xi_2 \ZZ + \cdots + \xi_n \ZZ \\
	& = \left \{ \sum_{i}^n x_i \xi_i : x_i \in \ZZ, \xi_i \in X \right \}.
\end{align*}
\end{defn}

\begin{thm}
A \emph{quadratic order} $\OO_\Delta$ of $\QQ(\sqrt\Delta)$ is a sub-ring of the quadratic integers of $\QQ(\sqrt\Delta)$ containing 1.  Following \cite[p.81]{Jacobson2009}, we write $\OO_\Delta$ as
\[
	\left[ 1, \frac{\Delta + \sqrt{\Delta}}{2} \right] = [1, f\omega_0].
\]
The maximal order $\OO_\Delta = [1, \omega_0]$ of $\QQ(\sqrt\Delta_0)$ is the ring of all quadratic integers in $\QQ(\sqrt\Delta_0)$.  This order is maximal since any other order $\OO = [1, f\omega_0]$ is a sub-ring of $\OO_\Delta$. 
\end{thm}


%%%%%%%%%%
% IDEALS %
%%%%%%%%%%
\bigbreak
\section{Ideals of $\OO_\Delta$}

\begin{defn}
An \emph{ideal} $\mathfrak a$ is an additive subgroup of an order $\OO$ with the property that for any $a \in \mathfrak a$ and $\xi \in \OO$, it holds that $\xi a$ and $a \xi$ are both elements of the ideal $\mathfrak a$.
\end{defn}

For any $\alpha, \beta \in \OO_\Delta$, the set $\{x \alpha + y \beta : x, y \in \OO_\Delta\}$ is an ideal $\mathfrak a$ in the order $\OO_\Delta$.  We say that $\mathfrak a$ is generated by $\alpha$ and $\beta$ and denote it $(\alpha, \beta)$.  Every ideal $\mathfrak a$ of a quadratic order $\OO_\Delta$ can be represented by at most two generators \cite{Cohn1980}, but some can be represented by only a single generator. An ideal represented by a single generator, $\alpha \in \OO_\Delta$, is denoted $(\alpha)$ and is called \emph{principal} \cite[p.87]{Jacobson2009}.

\begin{thm}
\label{thm:idealZModule}
When $\OO_\Delta = [1, f\omega_0]$ is the order of a quadratic field, a non-zero ideal $\mathfrak a$ of $\OO_\Delta$ can be uniquely written as a two dimensional $\ZZ$-module 
\[
	\mathfrak a = s\left[a, \frac{b+\sqrt{\Delta}}{2} \right]
\]
for $s, a, b \in \ZZ, s > 0, a > 0$, $\gcd(a, b, (b^2-\Delta)/4a)=1$, $b^2 \equiv \Delta \pmod{4a}$, and $b$ is unique $\bmod ~2a$ \cite[p.13]{Jacobson1999}. When $s = 1$, the ideal $\mathfrak a$ is \emph{primitive}.
\end{thm}

The \emph{identity} ideal is $\OO_\Delta = [1, f\omega_0]$ since $\mathfrak a = \mathfrak a \OO_\Delta = \OO_\Delta \mathfrak a$ \cite{Cohn1980}. An ideal $\mathfrak a = s[a, (b+\sqrt{\Delta})/2]$ is \emph{invertible} if there exists an ideal $\mathfrak b$ such that $\mathfrak a \mathfrak b = \OO_\Delta$, and an ideal $\mathfrak b$ exists when $\gcd(a, b, (b^2-\Delta)/(4a)) = 1$ \cite[p.14]{Jacobson1999}. The inverse is given by \cite[pp.14,15]{Jacobson1999}
\[
	{\mathfrak a}^{-1} = \frac{s}{\mathcal N(\mathfrak a)} \left[a, \frac{-b+\sqrt{\Delta}}{2} \right]
\]
where $\mathcal N(\mathfrak a) = s^2a$ is the norm of $\mathfrak a$ and is multiplicative. When $\OO_\Delta$ is maximal, all ideals of $\OO_\Delta$ are invertible. 

Prime ideals allow us to easily create an ideal.  For a prime ideal $\mathfrak p \in \OO_\Delta$ it can be shown (\cite[p.19]{Jacobson1999}) that $\mathfrak p \cap \ZZ = p\ZZ$ for some prime integer $p \in \ZZ$. Let
\[
	\mathfrak p = s\left[a, \frac{b + \sqrt{\Delta}}{2}\right]
\]
and it follows that either $s=p$ and $a=1$, or $s=1$ and $a=p$.  In the first case $\mathfrak p = p\OO_\Delta$, while in the second case $b = \pm \sqrt{\Delta} \pmod p$ and $\mathfrak p = [p, (b + \sqrt{\Delta})/2]$ when $4p ~|~ b^2 - \Delta$.  Algorithm \ref{alg:prime} gives pseudo-code.

% PRIME
\bigbreak
\begin{algorithm}[h]
\caption{Prime Ideal}
\label{alg:prime}
\begin{algorithmic}[1]
\REQUIRE A prime integer $p \in \ZZ$.
\ENSURE A representative $\mathfrak p = [p, (b+\sqrt\Delta)/2]$ such that $\mathfrak p$ is a prime ideal if one exists.
\STATE $b \gets \textrm{the positive } \sqrt\Delta \pmod p$
\IF {$4p ~|~ b^2-\Delta$}
	\RETURN $[p, (b+\sqrt\Delta)/2]$
\ENDIF
\STATE $b \gets p-b$
\IF {$4p ~|~ b^2-\Delta$}
	\RETURN $[p, (b+\sqrt\Delta)/2]$
\ENDIF
\RETURN none
\end{algorithmic}
\end{algorithm}

Two ideals $\mathfrak a$ and $\mathfrak b$ are \emph{equivalent} if there exists $\alpha, \beta \in \OO_\Delta$ such that $\alpha \beta \neq 0$ and $(\alpha)\mathfrak a = (\beta) \mathfrak b$ \cite[p.88]{Jacobson2009}. Following \cite[p.88]{Jacobson2009}, we denote by $[\mathfrak a]$ the \emph{ideal class} of all ideals equivalent to $\mathfrak a$. 


%%%%%%%%%%%%%%%%%%%%%
% IDEAL CLASS GROUP %
%%%%%%%%%%%%%%%%%%%%%
\bigbreak
\section{Ideal Class Group}

Recall that two ideals $\mathfrak a$ and $\mathfrak b$ are equivalent if there exists principal ideals $(\alpha)$ and $(\beta)$ such that $(\alpha)\mathfrak a = (\beta)\mathfrak b$.  An ideal class $[\mathfrak a]$ is the set of all ideals that are equivalent to $\mathfrak a$. As such, an ideal $\mathfrak a$ is a \emph{representative} for the ideal class $[\mathfrak a]$. The \emph{ideal class group}, $Cl_\Delta$, is the set of all equivalence classes of invertible $\OO$-ideals, with the group operation defined as the product of class representatives.

Our implementation of ideal arithmetic represents an ideal as $\mathfrak a = [a, (b + \sqrt\Delta)/2]$ where $\mathfrak a$ is primitive.  Additionally, we carry around the value $c = (b^2 - \Delta)/4a$.  We represent the class group by the discriminant $\Delta$. Since an ideal class contains an infinitude of ideals, we work with reduced representatives.  This also makes arithmetic faster, since the size of generators are typically smaller.

In Subsection \ref{subsec:reduction}, we state what it is for an ideal to be in reduced form, and in subsection \ref{subsec:idealMultiply}, we show how to multiply two reduced class representatives. In Subsection \ref{subsec:nucomp} we discuss how to perform multiplication such that the result is a reduced or almost reduced representative, and then extend this to the case of computing the square (\ref{subsec:nudupl}) and cube (\ref{subsec:nucube}) of an ideal class.  


%%%%%%%%%%%%%
% REDUCTION %
%%%%%%%%%%%%%
\subsection{Reduced Representatives}
\label{subsec:reduction}

\begin{defn}
A primitive ideal $\mathfrak{a} = [a, (b+\sqrt{\Delta})/2]$ with $\Delta < 0$ is \emph{reduced} when $-a < b \le a < c$ or when $0 \le b \le a = c$ for $c = (b^2 - \Delta)/4a$ \cite[p.241]{Crandall2000}.
\end{defn}

% REDUCE
\begin{algorithm}[h]
\caption{Ideal Reduction}
\label{alg:reduce}
\begin{algorithmic}[1]
\REQUIRE An ideal class representative $\mathfrak a_1 = [a_1, (b_1+\sqrt\Delta)/2]$ and $c_1 = ({b_1}^2 - \Delta)/4a_1$.
\ENSURE A reduced representative $\mathfrak a = [a, (b+\sqrt\Delta)/2]$.
\STATE $(a, b, c) \gets (a_1, b_1, c_1)$
\WHILE {$a > c$ or $b > a$ or $b \le -a$}
	\IF {$a > c$}
		\STATE swap $a$ with $c$ and let $b \gets -b$
	\ENDIF
	\IF {$b > a$ or $b \le -a$}
		\STATE $b \gets b'$ such that $-a < b' \le a$ and $b' \equiv b \pmod{2a}$
		\STATE $c \gets (b^2-\Delta)/4a$
	\ENDIF
\ENDWHILE
\IF {$a=c$ and $b < 0$}
	\STATE $b \gets -b$
\ENDIF
\RETURN $[a, (b+\sqrt\Delta)/2]$
\end{algorithmic}
\end{algorithm}

In an imaginary quadratic field, every ideal class contains exactly one reduced ideal \cite[p.20]{Ramachandran2006}.  There are several algorithms to compute a reduced ideal, many of which are listed in \cite{Jacobson2006}.  Here we present the algorithm we use.  We adapt the work presented on \cite[p.90]{Jacobson2006} and \cite[p.99]{Jacobson2009}. If $\mathfrak a = [a, (b + \sqrt\Delta)/2]$ is a primitive ideal then 
\begin{equation}
\label{eq:idealSwapNorm}
	\mathfrak b = \left[ -\mathcal N((b + \sqrt\Delta)/2)/a, -(b - \sqrt\Delta)/2 \right]
\end{equation}
is also a primitive ideal, since we can verify that
\[
	\left(-(b - \sqrt\Delta)/2 \right) \mathfrak a = (a) \mathfrak b.
\]
Simplifying Equation \ref{eq:idealSwapNorm} we get
\[
	\mathfrak b = \left[ \frac{b^2-\Delta}{4a}, \frac{-b + \sqrt\Delta}{2} \right].
\]
Since $c = (b^2 - \Delta)/4a$ we have
\begin{equation}
\label{eq:idealSwapAC}
	\mathfrak b = \left[ c, \frac{-b + \sqrt\Delta}{2} \right].
\end{equation}
As such, the first step is if $a > c$, we can reduce $a$ by setting $\mathfrak a = [c, (-b + \sqrt\Delta)/2]$.  Since $b$ is unique $\bmod{~2a}$, we can also reduce $b \bmod{2a}$.  We repeat these steps while $\mathfrak a$ is not reduced.  In the case that $a = c$, we use the absolute value of $b$, since by Equation \ref{eq:idealSwapAC} the ideals $[a, (b + \sqrt\Delta)/2]$ and $[c, (-b+\sqrt\Delta)/2]$ are equivalent.  Pseudo-code is given in Algorithm \ref{alg:reduce}.


%%%%%%%%%%%%%%%%%%
% MULTIPLICATION %
%%%%%%%%%%%%%%%%%%
\subsection{Multiplication of Ideal Classes}
\label{subsec:idealMultiply}

The ideal class group operation is multiplication of ideal class representatives. Given two representative ideals $\mathfrak a = [a_1, (b_1 + \sqrt{\Delta})/2]$ and $\mathfrak b = [a_2, (b_2 + \sqrt{\Delta})/2]$ in reduced form, the (non-reduced) product $\mathfrak a \mathfrak b$ can computed using
\begin{align}
	c_2 & = ({b_2}^2-\Delta)/4a_2, \\
	s & = \gcd(a_1, a_2, (b_1+b_2)/2) = Ya_1 + Va_2 + W(b_1+b_2)/2,    \label{eq:idealProductS} \\
	U & = (V(b_1-b_2)/2 - Wc_2) \bmod{(a_1/s)},                        \label{eq:idealProductU} \\
	a & = (a_1a_2)/s^2,                                                \label{eq:idealProductA} \\
	b & = (b_2 + 2Ua_2/s) \bmod{2a},                                   \label{eq:idealProductB} \\
	\mathfrak a \mathfrak b & = s\left[a, \frac{b + \sqrt{\Delta}}{2}\right].
\end{align}

The remainder of this subsection is used to derive the above equations.  We adapt much of the presentation given in \cite[pp.117,118]{Jacobson2009}. Component-wise multiplication of $\mathfrak a$ and $\mathfrak b$ give us
\begin{equation}
\label{eq:productExpanded}
\mathfrak{a} \mathfrak{b} =
\left[ a_1a_2, \frac{a_1b_2 + a_1\sqrt{\Delta}}{2}, \frac{a_2b_1 + a_2\sqrt{\Delta}}{2}, \frac{b_1b_2 + (b_1+b_2)\sqrt{\Delta} + \Delta}{4} \right].
\end{equation}

\noindent
By the multiplicative property of the norm we have
\begin{align*}
	& N(\mathfrak{a}\mathfrak{b}) = s^2a = N(\mathfrak{a})N(\mathfrak{b}) = a_1 a_2 \\
	\Rightarrow~ & a = \frac{a_1a_2}{s^2},
\end{align*}
which gives us Equation \ref{eq:idealProductA}. Now, by the second term of equation \eqref{eq:productExpanded} we know that $(a_1b_2 + a_1\sqrt{\Delta})/2 \in \mathfrak{a}\mathfrak{b}$.  It follows that there is some $x,y \in \ZZ$ such that
\[
	\frac{a_1b_2 + a_1\sqrt{\Delta}}{2} = xsa + ys\left(\frac{b+\sqrt{\Delta}}{2}\right).
\]
Equating irrational parts we have
\begin{equation*}
	\frac{a_1\sqrt{\Delta}}{2} = \frac{ys\sqrt{\Delta}}{2}.
\end{equation*}
\noindent
Hence, $s ~|~ a_1$.  Similarly, by the third and fourth terms of equation \eqref{eq:productExpanded} we have $(a_2b_1+a_2\sqrt{\Delta})/2 \in \mathfrak{a}\mathfrak{b}$, which implies that $s~|~a_2$, and $(b_1b_2 + (b_1+b_2)\sqrt{\Delta} + \Delta)/4 \in \mathfrak{a}\mathfrak{b}$, which implies that $s~|~(b_1+b_2)/2$. 

By the second generator, $s(b+\sqrt\Delta)/2$, of $\mathfrak{a}\mathfrak{b}$ and the entire right hand side of equation \eqref{eq:productExpanded} there exists $X, Y, V, W \in \ZZ$ such that
\[
\frac{sb+s\sqrt\Delta}{2} = Xa_1a_2 + Y\frac{a_1b_2+a_1\sqrt\Delta}{2} + V\frac{a_2b_1 + a_2\sqrt{\Delta}}{2} + W\frac{b_1b_2 + (b_1+b_2)\sqrt{\Delta} + \Delta}{4}.
\]
Grouping rational and irrational parts, we have
\begin{equation}
\label{eq:productSecond}
\frac{sb+s\sqrt\Delta}{2} = \left( Xa_1a_2 + Y\frac{a_1b_2}{2} + V\frac{a_2b_1}{2} + W\frac{b_1b_2 + \Delta}{4} \right) + \left(Y\frac{a_1}{2} + V\frac{a_2}{2} + W\frac{b_1+b_2}{4}\right)\sqrt\Delta. 
\end{equation}

\noindent
Again, by equating irrational parts we have
\begin{align}
	\frac{s\sqrt\Delta}{2} & = \left(Y\frac{a_1}{2} + V\frac{a_2}{2} + W\frac{b_1+b_2}{4}\right)\sqrt\Delta \nonumber \\
	s & = Ya_1 + Va_2 + W\frac{b_1+b_2}{2}, \label{eq:sAsGCD}
\end{align}
which is the same as Equation \ref{eq:idealProductS}.  Since $s$ divides each of $a_1, a_2,$ and $(b_1+b_2)/2$, we have that $s = \gcd(a_1, a_2, (b_1+b_2)/2)$.

It remains to compute $b \pmod{2a}$.  Recall that $a = a_1a_2/s^2$.  This time, by equating the rational parts of \eqref{eq:productSecond} we have:
\begin{align}
	\frac{sb}{2} & = Xa_1a_2 + Y\frac{a_1b_2}{2} + V\frac{a_2b_1}{2} + W\frac{b_1b_2 + \Delta}{4} \nonumber \\
	b & = 2X\frac{a_1a_2}{s} + Y\frac{a_1b_2}{s} + V\frac{a_2b_1}{s} + W\frac{b_1b_2 + \Delta}{2s} \nonumber \\
	b & \equiv Y\frac{a_1b_2}{s} + V\frac{a_2b_1}{s} + W\frac{b_1b_2 + \Delta}{2s} \pmod{2a} \label{eq:bMod2a}
\end{align}

\noindent
This gives us $b$.  However, we can rewrite \eqref{eq:bMod2a} with fewer multiplies and divides.  By equation \eqref{eq:sAsGCD}, we have
\begin{align*}
	s & = Ya_1 + Va_2 + W\frac{b_1+b_2}{2} \\
	1 & = Y\frac{a_1}{s} + V\frac{a_2}{s} + W\frac{b_1+b_2}{2s} \\
	Y\frac{a_1}{s} & = 1 - V\frac{a_2}{s} - W\frac{b_1+b_2}{2s}.
\end{align*}

\noindent
Substituting into equation \eqref{eq:bMod2a} we get
\begin{alignat*}{2}
	b & \equiv b_2(1-V\frac{a_2}{s} - W\frac{b_1+b_2}{2s}) + V\frac{a_2b_1}{s} + W\frac{b_1b_2 + \Delta}{2s} && \pmod{2a} \\
	& \equiv b_2 - V\frac{a_2b_2}{s} - W\frac{b_1b_2+{b_2}^2}{2s} + V\frac{a_2b_1}{s} + W\frac{b_1b_2 + \Delta}{2s} && \pmod{2a} \\
	& \equiv b_2 + V\frac{a_2(b_1-b_2)}{s} + W\frac{\Delta - {b_2}^2}{2s} && \pmod{2a} \\
	& \equiv b_2 + V\frac{2a_2(b_1-b_2)}{2s} + W\frac{2a_2(\Delta - {b_2}^2)}{2a_2 \cdot 2s} && \pmod{2a} \\
	& \equiv b_2 + \frac{2a_2}{s} \left( V\frac{b_1-b_2}{2} + W\frac{\Delta - {b_2}^2}{4a_2} \right) && \pmod{2a}.
\end{alignat*}
Let $c_2 = ({b_2}^2 - \Delta)/4a_2$ and $U = (V(b_1-b_2)/2 - Wc_2) \bmod{(a_1/s)}$ and we have
\[
	b \equiv b_2 + \frac{2a_2}{s} U \pmod{2a},
\]
which completes the derivation of Equation \ref{eq:idealProductB}.  Note that the product ideal $\mathfrak a \mathfrak b$ is not a reduced representative and that the storage needed can be as much as twice that of the ideal factors $\mathfrak a$ and $\mathfrak b$.


%%%%%%%%%%
% NUCOMP %
%%%%%%%%%%
\subsection{Fast Ideal Multiplication (NUCOMP)}
\label{subsec:nucomp}

Shanks \cite{Shanks1989} gives an algorithm for multiplying two ideal class representatives such that their product is reduced or almost reduced.  The algorithm is known as NUCOMP and stands for ``New COMPosition''.  This algorithm is often faster in practice as the intermediate numbers are smaller and the final product requires fewer (often no) applications of the reduction operator to be converted to reduced form. The description of NUCOMP provided here is a high level description of the algorithm based on \cite[pp.119-123]{Jacobson2009}.

Equations \ref{eq:idealProductS}, \ref{eq:idealProductA}, and \ref{eq:idealProductB} from the previous subsection give a solution to the ideal product $\mathfrak a \mathfrak b = s[a, (b+\sqrt\Delta)/2]$.  We begin with the observation (\cite[p.119]{Jacobson2009}) that the fraction $(b/2)/a$ is roughly equal to the fraction $sU / a_1$ where $U$ is given by Equation \ref{eq:idealProductU}:
\[
	\frac{b}{2a} = \frac{b_2 + 2Ua_2/s}{2a_1a_2/s^2} 
	= \frac{s^2 b_2+s2Ua_2}{2a_1a_2}
	= \frac{s^2b_2}{2a_1a_2} + \frac{sU}{a_1}
	\approx \frac{sU}{a_1}.
\]
Following \cite[pp.120-121]{Jacobson2009}, we develop the simple continued fraction expansion of $sU/a_1 = \langle q_0, q_1, \dots, q_i, \phi_{i+1} \rangle$ using the recurrences
\begin{align*}
	q_i &= \floor{R_{i-2} ~/~ R_{i-1}} \\
	R_i &= R_{i-2} - q_i R_{i-1} \\
	C_i &= C_{i-2} - q_i C_{i-1}
\end{align*}
until we have $R_i$ and $R_{i-1}$ such that
\begin{equation}
\label{eq:nucompBound}
	R_i < \sqrt{a_1/a_2} ~ |\Delta/4|^{1/4} < R_{i-1}.
\end{equation}
Initial values for the recurrence are given by
\[
	\matrixtt{R_{-2}}{R_{-1}}{C_{-2}}{C_{-1}} = \matrixtt{sU}{a_1}{-1}{0}.
\]
We then compute
\begin{align*}
	M_1 &= \frac{R_i a_2 + sC_i(b_1-b_2)/2}{a_1}, \\
	M_2 &= \frac{R_i (b_1+b_2)/2 - s C_i c_2}{a_1}, \\
	a &= (-1)^{i+1} (R_i M_1  - C_i M_2), \\
	b &= \left( \frac{2(R_i a_2 /s - C_{i-1} a)}{C_i} - b_2 \right) \bmod 2a
\end{align*}
for the reduced or almost reduced product $\mathfrak a \mathfrak b = [a, (b + \sqrt\Delta)/2]$.  Note that this procedure assumes that $\mathcal N(\mathfrak a) \ge \mathcal N(\mathfrak b)$ and that if $a_1 < \sqrt{a_1/a_2} ~ |\Delta/4|^{1/4}$ then $R_{-1}$ and $R_{-2}$ satisfy Equation \ref{eq:nucompBound} and we compute the product $\mathfrak a \mathfrak b$ as in the previous subsection without expanding the simple continued fraction $sU/a_1$.

Our implementation of fast ideal multiplication includes some optimizations on the above technique.  Equation \ref{eq:idealProductS} requires that we compute $\gcd(a_1, a_2, (b_1 + b_2)/2)$. Since $\gcd(a_1, a_2)$ is often equal to 1, we only need compute $\gcd(a_1, a_2, (b_1 + b_2)/2)$ when $\gcd(a_1, a_2) \neq 1$. Also, by Equation \ref{eq:idealProductS}, we have that $s ~|~ a_1$ and $s ~|~ a_2$, so we reduce both $a_1$ and $a_2$ by $s$ throughout. Finally, the first coefficient, $Y$, from Equation \ref{eq:idealProductS} is never used, and so we do not compute it. We give pseudo-code in Algorithm \ref{alg:nucomp}. 

% NUCOMP
\begin{algorithm}[h]
\caption{NUCOMP -- Fast Ideal Multiplication. Based on \cite[pp.441-443]{Jacobson2009}.}
\label{alg:nucomp}
\begin{algorithmic}[1]
\REQUIRE Reduced representatives $\mathfrak a = [a_1, (b_1+\sqrt\Delta)/2]$, $\mathfrak b = [a_2, (b_2+\sqrt\Delta)/2]$ \\ with $c_1 = ({b_1}^2-\Delta)/4a_1$, $c_2 = ({b_2}^2-\Delta)/4a_2$, and discriminant $\Delta$.
\ENSURE A reduced or almost reduced representative $\mathfrak a \mathfrak b$.
\STATE ensure $\mathcal N(\mathfrak a) < \mathcal N(\mathfrak b)$ by swapping $\mathfrak a$ with $\mathfrak b$ if $a_1 < a_2$
\STATE compute $s'$ and $V'$ such that $s' = \gcd(a_1, a_2) = Y'a_1 + V'a_2$ for $s', Y', V' \in \ZZ$
\STATE $s \gets 1$
\STATE $U \gets V'(b_1 - b_2)/2 \bmod a_1$
\IF {$s' \neq 1$}
	\STATE compute $s, V$, and $W$ \\
	       such that $s = \gcd(s', (b_1 + b_2)/2) = Vs' + W(b_1 + b_2)/2$ for $V, W \in \ZZ$
	\STATE $(a_1, a_2) \gets (a_1/s, a_2/s)$
	\STATE $U \gets (VU - Wc_2) \bmod a_1$
\ENDIF
\IF {$a_1 < \sqrt{a_1/a_2} ~ |\Delta/4|^{1/4}$}
	\STATE $a \gets a_1a_2$
	\STATE $b \gets (2a_2U + b_2) \bmod{2a}$
	\RETURN $[a, (b+\sqrt\Delta)/2]$
\ENDIF
\STATE $\matrixtt{R_{-2}}{R_{-1}}{C_{-2}}{C_{-1}} \gets \matrixtt{U}{a_1}{-1}{0}$
\STATE find $i$ such that $R_i < \sqrt{a_1/a_2} ~ |\Delta/4|^{1/4} < R_{i-1}$ using the recurrences: \\
$q_i = \floor{R_{i-2} ~/~ R_{i-1}}$ \\
$R_i = R_{i-2}-q_i R_{i-1}$ \\
$C_i=C_{i-2}-q_i C_{i-1}$
\STATE $M_1 \gets (R_i a_2 + C_i(b_1-b_2)/2)/a_1$
\STATE $M_2 \gets (R_i (b_1+b_2)/2 -sC_i c_2)/a_1$
\STATE $a \gets (-1)^{i+1}(R_i M_1 - C_i M_2)$
\STATE $b \gets ((2(R_i a_2 - C_{i-1} a)/C_i) - b_2) \bmod{2a}$
\RETURN $[a, (b+\sqrt\Delta)/2]$
\end{algorithmic}
\end{algorithm}


%%%%%%%%%%
% NUDUPL %
%%%%%%%%%%
\subsection{Fast Ideal Squaring (NUDUPL)}
\label{subsec:nudupl}

When the two input ideals for multiplication are the same, as is the case when squaring, much of the arithmetic simplifies.  For reduced ideals $\mathfrak a = [a_1, (b_1 + \sqrt\Delta)/2]$ and $\mathfrak b = [a_2, (b_2 + \sqrt\Delta)/2]$, we have $a_1=a_2$ and $b_1=b_2$.  Equations \ref{eq:idealProductS} and \ref{eq:idealProductU} simplify to
\begin{align*}
	s &= \gcd(a_1, b_1) = Xa_1 + Yb_1 \\
	U &= -Yc_1 \bmod (a_1/s).
\end{align*}
We then compute the continued fraction expansion of $sU/a_1$, but on the bound
\[
	R_i < |\Delta/4|^{1/4} < R_{i-1}.
\]
Computing the ideal class representative simplifies as well -- we have
\begin{align*}
	M_1 &= R_i, \\
	M_2 &= \frac{R_i b_1 - sC_i c_1}{a_1}, \\
	a &= (-1)^{i+1}({R_i}^2 - C_i M_2), \\
	b &= \left(\frac{2(R_i a_1/s  - C_{i-1} a)}{C_i} - b_1 \right) \bmod{2a}.
\end{align*}
As before, the representative $[a, (b+\sqrt\Delta)/2]$ is either reduced or almost reduced.  \break Pseudo-code for our implementation is given in algorithm \ref{alg:nudupl}.

% NUDUPL.
\begin{algorithm}[h]
\caption{NUDUPL -- Fast Ideal Squaring.}
\label{alg:nudupl}
\begin{algorithmic}[1]
\REQUIRE Reduced representative $\mathfrak a = [a_1, (b_1+\sqrt\Delta)/2]$ \\
         with $c_1 = ({b_1}^2-\Delta)/4a_1$ and discriminant $\Delta$.
\ENSURE A reduced or almost reduced representative $\mathfrak a^2$.
\STATE compute $s$ and $Y$ such that $s = \gcd(a_1, b_1) = Xa_1 + Yb_1$ for $s,X,Y \in \ZZ$
\STATE $a_1 \gets a_1/s$
\STATE $U \gets -Yc_1 \bmod a_1$
\IF {$a_1 < |\Delta/4|^{1/4}$}
	\STATE $a \gets {a_1}^2$
	\STATE $b \gets (2Ua_1 + b_1) \bmod 2a$
	\RETURN $[a, (b + \sqrt\Delta)/2]$
\ENDIF
\STATE $\matrixtt{R_{-2}}{R_{-1}}{C_{-2}}{C_{-1}} \gets \matrixtt{U}{a_1}{-1}{0}$
\STATE Find $i$ such that $R_i < |\Delta/4|^{1/4} < R_{i-1}$ using the recurrences: \\
       $q_i = \floor{R_{i-2}/R_{i-1}}$ \\
       $R_i = R_{i-2}-q_i R_{i-1}$ \\
       $C_i=C_{i-2}-q_i C_{i-1}$
\STATE $M_2 \gets (R_i b_1 -sC_i c_1)/a_1$
\STATE $a \gets (-1)^{i+1}({R_i}^2 - C_i M_2)$
\STATE $b \gets (2(R_i a_1 + C_{i-1} a)/C_i) \bmod{2a}$
\RETURN $[a, (b+\sqrt\Delta)/2]$
\end{algorithmic}
\end{algorithm}


%%%%%%%%%%
% NUCUBE %
%%%%%%%%%%
\subsection{Fast Ideal Cubing (NUCUBE)}\label{subsec:nucube}

When we consider binary-ternary representations of exponents, cubing is required.  In general, if we want to compute ${\mathfrak a}^3$ for an ideal class representative $\mathfrak a = [a_1, (b_1+\sqrt\Delta)/2]$, we can take advantage of the simplification that happens when expanding the computation of ${\mathfrak a}^2 \mathfrak a$.  Here we provide a high level description of a technique for cubing based on similar ideas to NUCOMP and NUDUPL, namely that of computing the quotients of a continued fraction expansion.  A detailed description and analysis of this technique can be found in \cite{Imbert2010}.

Similar to ideal squaring, we compute integers $s'$ and $Y'$ such that
\[
s' = \gcd(a_1, b_1) = X'a_1 + Y'b_1.
\]
Note that $X'$ is unused. If $s' \neq 1$ we compute
\[
s = \gcd(s'a_1, {b_1}^2 - a_1c_1) = Xs'a_1 + Y({b_1}^2 - a_1c_1)
\]
for $s, X, Y \in \ZZ$.  If $s' = 1$ then let $s = 1$ too.  We then compute $U$ using
\[
U = \begin{cases}
		Y'c_1(Y'(b_1 - Y'c_1a_1) - 2) \bmod {a_1}^2 & \textrm{ if } s' = 1 \\
		-c_1(XY'a_1+Yb_1) \bmod {a_1}^2/s & \textrm{ otherwise.}
    \end{cases}
\]
We then develop the simple continued fraction expansion of $sU/{a_1}^2$ until
\[
	R_i < \sqrt{a_1}|\Delta/4|^{1/4} < R_{i-1}.
\]
Finally, we can compute the reduced or almost reduced representative $\mathfrak a^3 = [a, (b + \sqrt\Delta)/2]$ using the equations
\begin{align*}
	M_1 &= \frac{(R_ia_1 + C_iUa_1)}{{a_1}^2}, \\
	M_2 &= \frac{R_i(b_1 + Ua_1) - sC_ic_1}{{a_1}^2}, \\
	a &= (-1)^{i+1} R_i M_1 - C_i M_2, \\
	b &= \left( \frac{2(R_ia_1/s - C_{i-1}a)}{C_i} - b_1 \right) \bmod 2a.
\end{align*}

As always, the ideal $[a, (b + \sqrt\Delta)/2]$ is reduced or almost reduced. Pseudo-code for our implementation of fast ideal cubing is given in Algorithm \ref{alg:nucube}.

% NUCUBE
\begin{algorithm}[h]
\caption{NUCUBE -- Fast Ideal Cubing. Adapted from \cite[p.26]{Imbert2010}.}
\label{alg:nucube}
\begin{algorithmic}[1]
\REQUIRE A reduced representative $\mathfrak a = [a_1, (b_1+\sqrt\Delta)/2]$.
\ENSURE A reduced or almost reduced representative $\mathfrak a^3$.
\STATE compute $s'$ and $Y'$ such that $s' = \gcd(a_1, b_1) = X'a_1 + Y'b_1$ for $s', X', Y' \in \ZZ$
\IF{$s' = 1$}
	\STATE $s \gets 1$
	\STATE $U \gets Y'c_1(Y'(b_1 - Y'c_1a_1) - 2) \bmod {a_1}^2$
\ELSE
	\STATE compute $s, X$, and $Y$ such that $s = \gcd(s'a_1, {b_1}^2 - a_1c_1) = Xs'a_1 + Y({b_1}^2 - a_1c_1)$ for $s, X, Y \in \ZZ$
	\STATE $U \gets -c_1(XY'a_1+Yb_1) \bmod {a_1}^2/s$
\ENDIF
\IF {${a_1}^2/s < \sqrt{a_1} ~ |\Delta/4|^{1/4}$}
	\STATE $a \gets {a_1}^3/s^2$
	\STATE $b \gets (b_1 + 2Ua_1/s) \bmod 2a$
	\RETURN $[a, (b+\sqrt\Delta)/2]$
\ENDIF
\STATE $\matrixtt{R_{-2}}{R_{-1}}{C_{-2}}{C_{-1}} \gets \matrixtt{U}{({a_1}^2/s)}{-1}{0}$
\STATE Find $i$ such that $R_i < \sqrt{a_1} |\Delta/4|^{1/4} < R_{i-1}$ using the recurrences: \\
       $q_i = \floor{R_{i-2}/R_{i-1}}$ \\
       $R_i = R_{i-2}-q_i R_{i-1}$ \\
       $C_i=C_{i-2}-q_i C_{i-1}$
\STATE $M_1 \gets (R_ia_1 + C_iUa_1) / {a_1}^2$
\STATE $M_2 \gets (R_i(b_1 + Ua_1) - sC_ic_1) / {a_1}^2$
\STATE $a \gets (-1)^{i+1} R_i M_1 - C_i M_2$
\STATE $b \gets (2(R_ia_1/s - C_{i-1}a)/C_i - b_1) \bmod 2a$
\RETURN $[a, (b+\sqrt\Delta)/2]$
\end{algorithmic}
\end{algorithm}

In the next chapter, we'll discuss some exponentiation techniques that make use of the ideal arithmetic presented in this chapter, namely fast multiplication, squaring, and cubing.

\end{document}
