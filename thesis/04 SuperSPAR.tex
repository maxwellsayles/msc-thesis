\documentclass{ucalgthes1}   
\usepackage[letterpaper,top=1in, bottom=1.22in, left=1.40in, right=0.850in]{geometry}
\usepackage{fancyhdr}
\fancyhead{}
\fancyfoot{}
\renewcommand{\headrulewidth}{0pt}
\fancyhead[RO,LE]{\thepage}  
\usepackage{hyperref}

\usepackage{algorithm}
\usepackage{algorithmic}
\usepackage{amsfonts}
\usepackage{amsmath}
\usepackage{amssymb}
\usepackage{amsthm}
\usepackage{comment}
\usepackage{float}
\usepackage{graphics}

\theoremstyle{plain}
\newtheorem{thm}{Theorem}[section]
\newtheorem{lemma}[thm]{Lemma}
\newtheorem{prop}[thm]{Proposition}
\newtheorem{cor}[thm]{Corollary}
\theoremstyle{definition}
\newtheorem{defn}[thm]{Definition}

\usepackage{eqparbox}
\renewcommand{\algorithmiccomment}[1]{\hfill\eqparbox{COMMENT}{#1}}
\renewcommand{\algorithmicrequire}{\textbf{Input:}}
\renewcommand{\algorithmicensure}{\textbf{Output:}}

\DeclareMathOperator*{\argmax}{arg\,max}
\DeclareMathOperator*{\argmin}{arg\,min}
\newcommand{\CC}{\mathbb{C}}
\newcommand{\NN}{\mathbb{N}}
\newcommand{\RR}{\mathbb{R}}
\newcommand{\KK}{\mathbb{K}}
\newcommand{\MM}{\mathcal{M}}
\newcommand{\OO}{\mathcal{O}}
\newcommand{\ZZ}{\mathbb{Z}}
\newcommand{\QQ}{\mathbb{Q}}
\newcommand{\NP}{\textrm{NP}}
\newcommand{\RRgtz}{\mathbb{R}_{>0}}
\newcommand{\ZZgtz}{\mathbb{Z}_{>0}}
\newcommand{\ZZgez}{\mathbb{Z}_{\ge 0}}
\newcommand{\QQgtz}{\mathbb{Q}_{>0}}
\newcommand{\QQgez}{\mathbb{Q}_{\ge 0}}
\newcommand{\matrixto}[2]{\left[ \begin{array}{rr} #1 & #2 \end{array} \right]}
\newcommand{\matrixot}[2]{\left[ \begin{array}{r} #1 \\ #2 \end{array} \right]}
\newcommand{\matrixtt}[4]{\left[ \begin{array}{rr} #1 & #2 \\ #3 & #4 \end{array} \right]}
\newcommand{\ntoinfty}{\lim_{n \rightarrow \infty}}
\newcommand{\floor}[1]{\left\lfloor #1 \right\rfloor}
\newcommand{\ceil}[1]{\left\lceil #1 \right\rceil}
\newcommand{\amax}{a_\textrm{max}}
\newcommand{\bmax}{b_\textrm{max}}

\begin{document}

\setcounter{chapter}{3}
\chapter{SuperSPAR}
\label{chap:superspar}


- An example of ideal exponentiation \\
- Connection of Ambiguous Ideals to the Factorization of the Discriminant \\

%%%%%%%%
% SPAR %
%%%%%%%%
%\bigbreak
\section{SPAR}

\subsection{Complexity}


%%%%%%%%%%%%%%%%%%%
% PRIMORIAL STEPS %
%%%%%%%%%%%%%%%%%%%
%\bigbreak
\section{Primorial Steps}

\subsection{How is the primorial bound chosen}

\subsection{Bound on primes}

\subsection{Bound on prime power}

\subsection{How is this done in the original SPAR}

\subsection{How is this done for generic groups}

\subsection{How do we do it (empirically)}


%%%%%%%%%%%%%
% SUPERSPAR %
%%%%%%%%%%%%%
%\bigbreak
\section{SuperSPAR}

\subsection{Complexity}
\subsection{Analysis for generic groups}
\subsection{Analysis for class gorup of imaginary quadratic number fields}
\subsection{Comparison with original SPAR}


\end{document}

