\documentclass{ucalgthes1}   
\usepackage[letterpaper,top=1in, bottom=1.22in, left=1.40in, right=0.850in]{geometry}
\usepackage{fancyhdr}
\fancyhead{}
\fancyfoot{}
\renewcommand{\headrulewidth}{0pt}
\fancyhead[RO,LE]{\thepage}  
\usepackage{hyperref}

\usepackage{algorithm}
\usepackage{algorithmic}
\usepackage{amsfonts}
\usepackage{amssymb}
\usepackage{amsmath}
\usepackage{amsthm}
\usepackage{comment}
\usepackage{float}

\theoremstyle{plain}
\newtheorem{thm}{Theorem}[section]
\newtheorem{lemma}[thm]{Lemma}
\newtheorem{prop}[thm]{Proposition}
\newtheorem{cor}[thm]{Corollary}
\theoremstyle{definition}
\newtheorem{defn}[thm]{Definition}

\usepackage{eqparbox}
\renewcommand{\algorithmiccomment}[1]{\hfill\eqparbox{COMMENT}{#1}}
\renewcommand{\algorithmicrequire}{\textbf{Input:}}
\renewcommand{\algorithmicensure}{\textbf{Output:}}

\DeclareMathOperator*{\argmax}{arg\,max}
\DeclareMathOperator*{\argmin}{arg\,min}
\newcommand{\CC}{\mathbb{C}}
\newcommand{\NN}{\mathbb{N}}
\newcommand{\RR}{\mathbb{R}}
\newcommand{\KK}{\mathbb{K}}
\newcommand{\MM}{\mathcal{M}}
\newcommand{\OO}{\mathcal{O}}
\newcommand{\ZZ}{\mathbb{Z}}
\newcommand{\QQ}{\mathbb{Q}}
\newcommand{\NP}{\textrm{NP}}
\newcommand{\RRgtz}{\mathbb{R}_{>0}}
\newcommand{\ZZgtz}{\mathbb{Z}_{>0}}
\newcommand{\ZZgez}{\mathbb{Z}_{\ge 0}}
\newcommand{\QQgtz}{\mathbb{Q}_{>0}}
\newcommand{\QQgez}{\mathbb{Q}_{\ge 0}}
\newcommand{\matrixto}[2]{\left[ \begin{array}{rr} #1 & #2 \end{array} \right]}
\newcommand{\matrixot}[2]{\left[ \begin{array}{r} #1 \\ #2 \end{array} \right]}
\newcommand{\matrixtt}[4]{\left[ \begin{array}{rr} #1 & #2 \\ #3 & #4 \end{array} \right]}
\newcommand{\ntoinfty}{\lim_{n \rightarrow \infty}}
\newcommand{\floor}[1]{\left\lfloor #1 \right\rfloor}
\newcommand{\ceil}[1]{\left\lceil #1 \right\rceil}

\newcommand{\amax}{a_\textrm{max}}
\newcommand{\bmax}{b_\textrm{max}}

\begin{document}

\chapter*{Appendix A - Algorithms}

%%%%%%%%%%%%%%%%%%%%%%%%%%%%%%%%%%%%
% IDEAL CLASS CUBING (NON-REDUCED) %
%%%%%%%%%%%%%%%%%%%%%%%%%%%%%%%%%%%%
\begin{algorithm}[h]
\caption{Ideal Class Cubing (Non-Reduced)}
\label{alg:idealCube}
\begin{algorithmic}[1]
\REQUIRE Reduced ideal class representative $\mathfrak a = \left[a_1, (b_1 + \sqrt\Delta)/2\right]$ and $c_1 = ({b_1}^2-\Delta)/4a_1$
\ENSURE Non-reduced ideal class representative ${\mathfrak a}^3$
\STATE $s_1 \gets U_1a_1 + V_1b_1$ for $U_1,V_1 \in \ZZ$
\IF{$s_1 = 1$}
	\STATE $s_2 \gets 1$
	\STATE $N \gets a_1$
	\STATE $L \gets {a_1}^2$
	\STATE $K \gets c_1V_1(2-V_1(b_1-V_1a_1c_1)) \pmod L$
\ELSE
	\STATE $s_2 \gets U_2(s_1a_1) + V_2({b_1}^2-a_1c_1)$ for $U_2, V_2 \in \ZZ$
	\STATE $N \gets a_1/s_2$
	\STATE $L \gets Na_1$
	\STATE $K \gets c_1(U_2V_1a_1+V_2b_1) \pmod L$
\ENDIF
\STATE $T \gets NK$
\STATE $a \gets NL$
\STATE $b \gets b_1 - 2T$
\STATE $c \gets (s_2c_1+K(T-b))/L$ \COMMENT{optional}
\RETURN $[a, (b+\sqrt\Delta)/2]$ \COMMENT{optionally $c$}
\end{algorithmic}
\end{algorithm}




%%%%%%%%%%%%%%%%%%%%%%%%%%%
% COMPUTER REPRESENTATION %
%%%%%%%%%%%%%%%%%%%%%%%%%%%
\bigbreak
\section{Computer Representation}
\label{section:computerRepresentation}

Here we provide pseudo-code for operations in the ideal class group of imaginary quadratic number fields.  When algorithms \ref{alg:nucomp}, \ref{alg:nudupl}, and \ref{alg:nucube} specify to compute $s \leftarrow Va + Wb$ for some $V,W \in \ZZ$, the details of the computation are intentionally left out.  A sufficient method is to use the extended Euclidean algorithm.  Similarly, when these algorithms specify to compute $R_i$ to some bound using the recurrences in \eqref{eq:cf1} through \eqref{eq:cf3}, a bounded extended Euclidean algorithm for only one extended attribute suffices. In chapter 5 we discuss several variations of the extended Euclidean algorithm suitable to these computations and provide an empirical analysis of their performance on various bit sizes.

\bigbreak
\bigbreak


%%%%%%%%%%%%
% IDENTITY %
%%%%%%%%%%%%
\begin{comment}
\begin{algorithm}[h]
\caption{Identity}
\label{alg:identity}
\begin{algorithmic}[1]
\ENSURE A reduced representative for the class of principal ideals.
\IF {$\Delta \equiv 1 \pmod 4$}
	\STATE Let $b = 1$.
\ELSE
	\STATE Let $b = 0$.
\ENDIF
\RETURN $[1, (b+\sqrt\Delta)/2]$.
\end{algorithmic}
\end{algorithm}
\end{comment}

%%%%%%%%%%%
% INVERSE %
%%%%%%%%%%%
\begin{comment}
\begin{algorithm}[h]
\caption{Iverse}
\label{alg:inverse}
\begin{algorithmic}[1]
\REQUIRE A reduced representative $\mathfrak a = [a, (b+\sqrt\Delta)/2]$.
\ENSURE A reduced representative $\mathfrak a^{-1}$ such that $\mathfrak a \mathfrak a^{-1}$ is principal.
\RETURN $[a, (-b+\sqrt\Delta)/2]$.
\end{algorithmic}
\end{algorithm}
\end{comment}


%%%%%%%%%
% PRIME %
%%%%%%%%%
\begin{algorithm}[h]
\caption{Prime Ideal}
\label{alg:prime}
\begin{algorithmic}[1]
\REQUIRE A prime integer $p \in \ZZ$.
\ENSURE A representative $\mathfrak p = [p, (b+\sqrt\Delta)/2]$ such that $\mathfrak p$ is a prime ideal if one exists.
\STATE $b \gets \textrm{the positive } \sqrt\Delta \pmod p$
\IF {$4p ~|~ b^2-\Delta$}
	\RETURN $[p, (b+\sqrt\Delta)/2]$
\ENDIF
\STATE $b \gets p-b$
\IF {$4p ~|~ b^2-\Delta$}
	\RETURN $[p, (b+\sqrt\Delta)/2]$
\ENDIF
\RETURN none
\end{algorithmic}
\end{algorithm}


%%%%%%%%%%
% REDUCE %
%%%%%%%%%%
\begin{algorithm}[h]
\caption{Reduce}
\label{alg:reduce}
\begin{algorithmic}[1]
\REQUIRE An ideal class representative $\mathfrak a_1 = [a_1, (b_1+\sqrt\Delta)/2]$ and $c_1 = ({b_1}^2 - \Delta)/4a_1$.
\ENSURE A reduced representative $\mathfrak a = [a, (b+\sqrt\Delta)/2]$.
\STATE $(a, b, c) \gets (a_1, b_1, c_1)$
\WHILE {$a > c$ or $b > a$ or $b \le -a$}
	\IF {$a > c$}
		\STATE swap $a$ with $c$ and let $b \gets -b$
	\ENDIF
	\IF {$b > a$ or $b \le -a$}
		\STATE $b \gets b'$ such that $-a < b' \le a$ and $b' \equiv b \pmod{2a}$
		\STATE $c \gets (b^2-\Delta)/4a$
	\ENDIF
\ENDWHILE
\IF {$a=c$ and $b < 0$}
	\STATE $b \gets -b$
\ENDIF
\RETURN $[a, (b+\sqrt\Delta)/2]$
\end{algorithmic}
\end{algorithm}



%%%%%%%%%%%%%
% AMBIGUOUS %
%%%%%%%%%%%%%
\begin{comment}
\begin{algorithm}[h]
\caption{Is Ambiguous?}
\label{alg:ambiguous}
\begin{algorithmic}[1]
\REQUIRE A reduced representative $\mathfrak a = [a, (b+\sqrt\Delta)/2]$.
\ENSURE True if $\mathfrak a$ is ambiguous, False otherwise.
\IF {$a > 1$}
	\IF {$b = 0$ or $a = b$ or $a = c$}
		\RETURN True.
	\ENDIF
\ENDIF
\RETURN False.
\end{algorithmic}
\end{algorithm}
\end{comment}

%%%%%%%%%%
% NUCOMP %
%%%%%%%%%%
\begin{algorithm}[h]
\caption{NUCOMP}
\label{alg:nucomp}
\begin{algorithmic}[1]
\REQUIRE Reduced representatives $\mathfrak a = [a_1, (b_1+\sqrt\Delta)/2]$, $\mathfrak b = [a_2, (b_2+\sqrt\Delta)/2]$ with \break $c_1 = ({b_1}^2-\Delta)/4a_1$, $c_2 = ({b_2}^2-\Delta)/4a_2$, and discriminant $\Delta$.
\ENSURE A reduced or almost reduced representative $\mathfrak a \mathfrak b$.
\STATE $s \gets Ya_1 + Va_2 + W (b_1+b_2)/2$ for $Y, V, W \in \ZZ$
\STATE $U \gets V(b_1-b_2)/2 + Wc_2$
\STATE $\matrixtt{R_{-2}}{R_{-1}}{C_{-2}}{C_{-1}} \gets \matrixtt{sU}{a_1}{-1}{0}$
\STATE Find $i$ such that $R_i < \sqrt{a_1/a_2} ~ |\Delta/4|^{1/4} < R_{i-1}$ using the recurrences: \\
$q_i = \floor{R_{i-2}/R_{i-1}}$ \\
$R_i = R_{i-2}-q_i R_{i-1}$ \\
$C_i=C_{i-2}-q_i C_{i-1}$
\STATE $M_1 \gets (R_i a_2 + sC_i(b_1-b_2)/2)/a_1$
\STATE $M_2 \gets (R_i (b_1+b_2)/2 -sC_i c_2)/a_1$
\STATE $a \gets (-1)^{i+1}(R_i M_1 - C_i M_2)$
\STATE $b \gets ((R_i a_2/s + C_{i-1} a)/C_i) \bmod{2a}$
\RETURN $[a, (b+\sqrt\Delta)/2]$
\end{algorithmic}
\end{algorithm}

%%%%%%%%%%%
% NUDUPL %
%%%%%%%%%%%
\begin{algorithm}[h]
\caption{NUDUPL}
\label{alg:nudupl}
\begin{algorithmic}[1]
\REQUIRE Reduced representative $\mathfrak a = [a_1, (b_1+\sqrt\Delta)/2]$ with $c_1 = ({b_1}^2-\Delta)/4a_1$ and discriminant $\Delta$.
\ENSURE A reduced or almost reduced representative $\mathfrak a^2$.
\STATE $s \gets Xa_1 + Yb_1$ for $X,Y \in \ZZ$
\STATE $U \gets Yc_1$
\STATE $\matrixtt{R_{-2}}{R_{-1}}{C_{-2}}{C_{-1}} \gets \matrixtt{sU}{a_1}{-1}{0}$
\STATE Find $i$ such that $R_i < \sqrt{a_1/a_2} ~ |\Delta/4|^{1/4} < R_{i-1}$ using the recurrences: \\
$q_i = \floor{R_{i-2}/R_{i-1}}$ \\
$R_i = R_{i-2}-q_i R_{i-1}$ \\
$C_i=C_{i-2}-q_i C_{i-1}$
\STATE $M_2 \gets (R_i b_1 -sC_i c_1)/a_1$
\STATE $a \gets (-1)^{i+1}({R_i}^2 - C_i M_2)$
\STATE $b \gets ((R_i a_2/s + C_{i-1} a)/C_i) \bmod{2a}$
\RETURN $[a, (b+\sqrt\Delta)/2]$
\end{algorithmic}
\end{algorithm}

%%%%%%%%%%
% NUCUBE %
%%%%%%%%%%
\begin{algorithm}[h]
\caption{NUCUBE}
\label{alg:nucube}
\begin{algorithmic}[1]
\REQUIRE A reduced representative $\mathfrak a = [a_1, (b_1+\sqrt\Delta)/2]$.
\ENSURE A reduced or almost reduced representative $\mathfrak a^3$.
\STATE $s_1 \gets U_1a_1 + V_1b_1$ for $U_1,V_1 \in \ZZ$
\IF{$s_1 = 1$}
	\STATE $s_2 \gets 1$
	\STATE $N \gets a_1$
	\STATE $L \gets {a_1}^2$
	\STATE $K \gets c_1V_1(2-V_1(b_1-V_1a_1c_1)) \bmod L$
\ELSE
	\STATE $s_2 \gets U_2(s_1a_1) + V_2({b_1}^2-a_1c_1)$ for $U_2, V_2 \in \ZZ$
	\STATE $N \gets a_1/s_2$
	\STATE $L \gets Na_1$
	\STATE $K \gets c_1(U_2V_1a_1+V_2b_1) \bmod L$
\ENDIF
\IF {$L < \sqrt{a_1} ~ |\Delta/4|^{1/4}$}
	\STATE $a \gets NL$
	\STATE $b \gets b_1 - 2NK$
\ELSE
	\STATE $\matrixtt{R_{-1}}{R_0}{C_{-1}}{C_0} \gets \matrixtt{L}{K}{0}{-1}$
	\STATE Find $i$ such that $R_i < \sqrt{a_1} ~ |\Delta/4|^{1/4} < R_{i-1}$ using the recurrences:\\
	$q_i = \floor{R_{i-2}/R_{i-1}}$ \\
	$R_i = R_{i-2}-q_i R_{i-1}$ \\
	$C_i=C_{i-2}-q_i C_{i-1}$
	\STATE $b_2 \gets b_1 -2NK \pmod L$
	\STATE $M_1 \gets (NR_i + (b_2-b_1)C_i)/L$
	\STATE $M_2 \gets (R_i(b_1+b_2)+c_1s_2)/L$
	\STATE $a \gets (-1)^{i-1}(R_iM_1-C_iM_2)$
	\STATE $b \gets (NR_i + aC_{i-1})/C_i-b_1$
\ENDIF
\STATE $b \gets b \bmod 2a$
\STATE $c \gets (b^2-\Delta)/4a$ \COMMENT{Optional}
\RETURN $[a, (b+\sqrt\Delta)/2]$ \COMMENT{Optionally $c$}
\end{algorithmic}
\end{algorithm}





\begin{comment}
Given an ideal $\mathfrak a = [a, (b+\sqrt\Delta)/2]$, reduction is performed by repeated application of the reduction operator $\rho$ \cite[Lecture 2, Slide 5/21]{JacLecture} where
\[
	\rho(\mathfrak a) = \left[-\frac{N(\beta)}{a}, -\overline{\beta}\right]
\]
for
\begin{align*}
	\beta &= (b+\sqrt\Delta)/2 -qa \\
	q &= \begin{cases}
		\floor{\frac{b}{2a}} & \mbox{ if } \Delta < 0 \\
		\floor{\frac{b+\sqrt\Delta}{2a}} & \mbox{ if } \Delta > 0.
	\end{cases}
\end{align*}

\noindent
An ideal $\mathfrak a$ is reduced when $\mathfrak a$ is primitive and $N(\mathfrak a) < \sqrt{|\Delta|}/2$ \cite[Theorem 5.6 p.99, and Theorem 5.9 p.101]{Jac09}.  

The algorithm in \cite[pp.104-105]{Jac09} to produce a reduced ideal given $\mathfrak a = [a, (b+\sqrt\Delta)/2]$ is to compute the simple continued fraction expansion of $b/|a|$ by the Euclidean algorithm.  If $|a| < \sqrt{|\Delta|}/2$ then $\mathfrak a$ is reduced, so suppose $|a| > \sqrt{|\Delta|}/2$.  Let $R_{-2}=b$, $R_{-1}=|a|$, $C_{-2}=-1$, and $C_{-1} = 0$, then $b/|a| = \langle q_0, q_1, \dots, q_i, R_i/R_{i+1} \rangle$ is given by
\begin{align}
	q_i &= \floor{ R_{i-2}/R_{i-1} } \label{eq:cf1} \\
	R_i &= R_{i-2} - q_i R_{i-1} \label{eq:cf2} \\
	C_i &= C_{i-2} - q_i C_{i-1}. \label{eq:cf3}
\end{align}
Steps \eqref{eq:cf1} through \eqref{eq:cf3} are the Euclidean algorithm; as such, this terminates with $R_n = 0$ for some $n$.  Since $R_{-1} = |a| > \sqrt{|\Delta|}/2$ and $R_n = 0$, there must exist some integer $i$ such that $0 \le i \le n$ and 
\[
	R_i < \left(|a|\sqrt{|\Delta|}/2\right)^{1/2} = \sqrt{|a|} ~ {|\Delta/4|}^{1/4} < R_{i-1}.
\]
In the case of $\Delta>0$, the reduced form is given by
\[
	{\mathfrak a}_{i+2} = \left[ a_{i+1}, \frac{b_{i+1} + \sqrt\Delta}{2} \right]
\]
where
\begin{align*}
	a_{i+1} &= (-1)^{i+1} r\frac{ 4{R_i}^2 -\Delta{C_i}^2 }{|4a|}, &
	b_{i+1} &= r\frac{R_i + a_{i+1}C_{i-1}}{C_i}
\end{align*}
for
\[
	r = \begin{cases}
		1 & \mbox{ when } \Delta \equiv 0 \pmod 4 \\
		2 & \mbox{ when } \Delta \equiv 1 \pmod 4.
	\end{cases}
\]

\noindent
When $\Delta<0$, either ${\mathfrak a}_{i+2}$ is reduced or $\rho({\mathfrak a}_{i+2})$ is reduced.  See \cite[pp.~104-106]{Jac09} for additional details.  This use of a continued fraction expansion to reduce an ideal will be seen again in subsection \ref{subsec:nucomp} when we discuss Shanks' NUCOMP algorithm for computing the reduced or almost reduced product of two ideal class representatives.
\end{comment}


%%%%%%%%%%%%%%%%%%%%%%%%%%
% R2L NAF EXPONENTIATION %
%%%%%%%%%%%%%%%%%%%%%%%%%%
\begin{algorithm}[h]
\caption{Computes $g^n$ using right-to-left non-adjacent form. Joye \cite{Joye2000}}
\label{alg:nafR2LImmutable}
\begin{algorithmic}[1]
\REQUIRE $g \in G, n \in \ZZgez$
\STATE $c \gets 0$ \COMMENT{carry flag}
\STATE $T \gets g$ \COMMENT{invariant: $T = g^{2^i}$}
\STATE $R \gets 1_G$
\STATE $i \gets 0$
\WHILE {$n \ge 2^i$}
	\IF {$\floor{n/2^i}+c \equiv 1 \pmod 4$}
		\STATE $R \gets R \cdot T$
		\STATE $c \gets 0$
	\ELSIF {$\floor{n/2^i}+c \equiv 3 \pmod 4$}
		\STATE $R \gets R \cdot T^{-1}$
		\STATE $c \gets 1$
	\ENDIF
	\STATE $T \gets T^2$
	\STATE $i \gets i+1$
\ENDWHILE
\IF {$c=1$} \STATE $R \gets R \cdot T$ \ENDIF
\RETURN $R$
\end{algorithmic}
\end{algorithm}


%%%%%%%%%%%%%%%%%%%%%%%%%%%%%%
% EXPONENTIATE USING A CHAIN %
%%%%%%%%%%%%%%%%%%%%%%%%%%%%%%
\begin{algorithm}[h]
\caption{Computes $g^n$ given $n$ as a chained 2,3 partition. TODO: Cite something.}
\label{alg:expWithChain}
\begin{algorithmic}[1]
\REQUIRE $g \in G,$ 
$n = \sum_{i=1}^k s_i2^{a_i}3^{b_i},$ \\
$s_1,...,s_k \in \{-1, 1\},$ 
$0 \le a_1 \le ...\le a_k \in \ZZ,$ 
$0 \le b_1 \le ... \le b_k \in \ZZ.$
\STATE $i \gets 1$
\STATE $a \gets 0$ \COMMENT{current power of 2}
\STATE $b \gets 0$ \COMMENT{current power of 3}
\STATE $T \gets g$ \COMMENT{loop invariant: $T = g^{2^a 3^b}$}
\STATE $R \gets 1_G$
\WHILE {$i \le k$}
	\WHILE {$a < a_i$}
		\STATE $T \gets T^2, a \gets a + 1$
	\ENDWHILE
	\WHILE {$b < b_i$}
		\STATE $T \gets T^3, b \gets b + 1$
	\ENDWHILE
	\STATE $R \gets R \cdot T^{s_i}$ \COMMENT{multiply with $T$ or $T^{-1}$}
	\STATE $i \gets i + 1$
\ENDWHILE
\RETURN $R$
\end{algorithmic}
\end{algorithm}

%%%%%%%%%%%%%%%%%%%%%%%%%%%%
% EXPONENTIATE USING YAO'S %
%%%%%%%%%%%%%%%%%%%%%%%%%%%%
\begin{algorithm}[h]
\caption{Computes $g^n$ given $n$ in 2,3 representation. M\'{e}loni \cite[Section 3.2]{Meloni2009}.}
\label{alg:yaos}
\begin{algorithmic}[1]
\REQUIRE $g \in G,$ 
$n = \sum_{i=1}^k s_i2^{a_i}3^{b_i},$ \\
$s_1,...,s_k \in \{-1, 1\},$ 
$a_1 \ge ... \ge a_k \in \ZZgez,$ 
$b_1,...,b_k \in \ZZgez.$
\STATE $T_b \gets g^{3^b}$ for $0 \le b \le \max \{ b_1, ..., b_k \}$ \COMMENT{by repeated cubing}
\STATE $R \gets 1_G$
\STATE $i \gets 1$
\WHILE {$i < k$}
	\STATE $R \gets R \cdot {T_{b_i}}^{s_i}$ \COMMENT{multiply with $T_{b_i}$ or ${T_{b_i}}^{-1}$}
	\STATE $a_\Delta \gets a_i - a_{i+1}$
	\STATE $R \gets R ^ {2^{a_\Delta}}$ \COMMENT{by squaring $a_\Delta$ number of times}
	\STATE $i \gets i + 1$
\ENDWHILE
\STATE $R \gets R \cdot {T_{b_k}}^{s_k}$
\STATE $R \gets R ^ {2^{a_k}}$ \COMMENT{by squaring $a_k$ number of times}
\RETURN $R$
\end{algorithmic}
\end{algorithm}

%%%%%%%%%%%%%%%%%
% R2L 2,3 CHAIN %
%%%%%%%%%%%%%%%%%
\begin{algorithm}[h]
\caption{Computes a 2,3 strictly chained representation from low order to high order. Ciet \cite{Ciet2006}.}
\label{alg:rtolDbnsChain}
\begin{algorithmic}[1]
\REQUIRE $n \in \ZZgez$
%\ENSURE $n=\sum_{i=1}^k s_i2^{a_i}3^{b_i}$ as a strictly chained partition.
\STATE $(a, b) \gets (0, 0)$
\STATE $i \gets 1$
\WHILE {$n > 0$}
	\WHILE {$n \equiv 0 \pmod 2$} 
		\STATE $n \gets n / 2, a \gets a + 1$
	\ENDWHILE
	\WHILE {$n \equiv 0 \pmod 3$}
		\STATE $n \gets n / 3, b \gets b + 1$
	\ENDWHILE
	\IF {$n \equiv 1 \pmod 3$}
		\STATE $n \gets n - 1, s \gets 1$
	\ELSIF {$n \equiv 2 \pmod 3$}
		\STATE $n \gets n + 1, s \gets -1$
	\ENDIF
	\STATE $(s_i, a_i, b_i) \gets (s, a, b)$
	\STATE $i \gets i + 1$
\ENDWHILE
\STATE $k \gets i$
\RETURN $(a_1, b_1, s_1), ..., (a_k, b_k, s_k)$
\end{algorithmic}
\end{algorithm}

%%%%%%%%%%%%%%
% GREEDY L2R %
%%%%%%%%%%%%%%
\begin{algorithm}[h]
\caption{Greedy left to right representations. Berth\'{e} and Imbert \cite{Berthe2009}.}
\label{alg:greedyltor}
\begin{algorithmic}[1]
\REQUIRE $n, \amax, \bmax \in \{\ZZgez, +\infty\}$ \COMMENT{$+\infty$ for unbounded $a$ or $b$}
\STATE $L \gets \textrm{empty list}$
\WHILE {$n \ne 0$}
	\STATE $(a, b) \gets \argmin_{(c, d)} \left\{\left||n| - 2^c3^d \right| : 0 \le c \le \amax, 0 \le d \le \bmax \right\}$
	\STATE $s \gets -1 \textrm{ when } n < 0 \textrm{ and } 1 \textrm{ otherwise}$
	\STATE $\textrm{push }(s, a, b) \textrm{ onto the front of } L$
	\STATE $\textrm{optionally set } (\amax, \bmax) \gets (a, b) \textrm{ when a chain is desired}$
	\STATE $n \gets n - s2^a3^b$
\ENDWHILE
\RETURN $L$
\end{algorithmic}
\end{algorithm}

%%%%%%%%%%%%%
% DBNS TREE %
%%%%%%%%%%%%%
\begin{algorithm}[h]
\caption{Tree-Based Chained 2,3 Partitions. Doche and Habsieger \cite{Doche2008}.}
\label{alg:dbnsTree}
\begin{algorithmic}[1]
\REQUIRE $n \in \ZZgtz$
\STATE $T \gets$ a binary tree on the node $n$
\WHILE {no leaf is 1}
	\FORALL {leaf nodes $v \in T$}
		\STATE insert as a left child $(v - 1)$ with all factors of 2 and 3 removed
		\STATE insert as a right child $(v + 1)$ with all factors of 2 and 3 removed
	\ENDFOR
	\STATE discard any duplicate leaves
	\STATE discard all but the smallest $B$ leaves
\ENDWHILE
\RETURN the chained 2,3 partition generated by the path from the root to the first leaf node containing 1
\end{algorithmic}
\end{algorithm}



\end{document}
