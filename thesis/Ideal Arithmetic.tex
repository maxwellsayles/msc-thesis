\documentclass{ucalgthes1}   
\usepackage[letterpaper,top=1in, bottom=1.22in, left=1.40in, right=0.850in]{geometry}
\usepackage{fancyhdr}
\fancyhead{}
\fancyfoot{}
\renewcommand{\headrulewidth}{0pt}
\fancyhead[RO,LE]{\thepage}  
\usepackage{hyperref}

\usepackage{algorithmic}
\usepackage{amsfonts}
\usepackage{amssymb}
\usepackage{amsmath}
\usepackage{amsthm}
\usepackage{comment}

\newtheorem*{thm}{Theorem}
\newtheorem*{lem}{Lemma}
\newtheorem*{cor}{Corollary}
\theoremstyle{definition}
\newtheorem*{defn}{Definition}


\parindent 0ex

\newcommand{\CC}{\mathbb{C}}
\newcommand{\NN}{\mathbb{N}}
\newcommand{\RR}{\mathbb{R}}
\newcommand{\KK}{\mathbb{K}}
\newcommand{\MM}{\mathcal{M}}
\newcommand{\OO}{\mathcal{O}}
\newcommand{\ZZ}{\mathbb{Z}}
\newcommand{\QQ}{\mathbb{Q}}
\newcommand{\NP}{\textrm{NP}}
\newcommand{\RRgtz}{\mathbb{R}_{>0}}
\newcommand{\ZZgtz}{\mathbb{Z}_{>0}}
\newcommand{\ZZgez}{\mathbb{Z}_{\ge 0}}
\newcommand{\QQgtz}{\mathbb{Q}_{>0}}
\newcommand{\QQgez}{\mathbb{Q}_{\ge 0}}
\newcommand{\matrixot}[2]{\left( \begin{array}{r} #1 \\ #2 \end{array} \right)}
\newcommand{\matrixtt}[4]{\left( \begin{array}{rr} #1 & #2 \\ #3 & #4 \end{array} \right)}
\newcommand{\ntoinfty}{\lim_{n \rightarrow \infty}}
\newcommand{\floor}[1]{\left\lfloor #1 \right\rfloor}
\newcommand{\ceil}[1]{\left\lceil #1 \right\rceil}


\begin{document}


\setcounter{chapter}{1}
\chapter{Ideal Arithmetic}


The material in this section is covered in depth in TODO.  Here we provide as a reminder some number theoretic concepts relevant to ideal arithmetic as it applies to this thesis.


\bigbreak
A complex number $\alpha$ is called an \emph{algebraic number} if it is a root of a polynomial $f(x) \in \QQ[x]$.  Let $f'(x) = cf(x)$ for some $c \in \ZZ$ such that $f'(x) \in \ZZ[x]$.  Then $f'(x)$ has the same roots as $f(x)$.  If $f'(x) \in \ZZ[x]$ is a monic polynomial, then a root $\alpha$ is called an \emph{algebraic integer}.  Given an algebraic number $\alpha$, we create an algebraic number field as the extension field of $\QQ$ adjoined $\alpha$, denoted $\QQ(\alpha)$. Let $\KK$ be our algebraic number field:
\[
	\KK = \QQ(\alpha) = \left\{ \frac{f(\alpha)}{g(\alpha)} : f(x), g(x) \in \QQ[x]; g(\alpha) \ne 0 \right\}.
\]


\bigbreak
The rational numbers $\QQ$ are algebraic numbers of degree 1, i.e. $a/b \in \QQ$ is a root of $bx - a$.  Quadratic numbers are algebraic numbers of degree 2.  For example, given a polynomial $f(x) = ax^2 + bx + c$ where $f(x) \in \ZZ[x]$, the root $\alpha$ is given by the quadratic formula
\[
	\alpha = \frac{-b \pm \sqrt{b^2 - 4ac}}{2a}.
\]
If we let $\Delta = b^2 -4ac$ such that $\Delta \in \ZZ$ be the discriminant of $f(x)$, our quadratic number field is 
\[
	\KK = \QQ(\alpha) = \QQ(\sqrt{\Delta}) = \{u + v\sqrt{\Delta} : u,v \in \QQ\}.
\]
Notice that if $\Delta = f^2 \Delta_0$ where $\Delta_0$ is squarefree, then $\KK = \QQ(\Delta_0)$.  


\bigbreak
\section{Quadratic Integers}
An algebraic integer $\beta$ of $\KK = \QQ(\sqrt{\Delta_0})$ is $\beta = u+v \sqrt{\Delta_0}$ such that $\beta$ is the root of a monic quadratic polynomial $f(x) = x^2+bx+c \in \ZZ[x]$ for some $b,c \in \ZZ$.  Let us characterize the algebraic integers of $\KK = \QQ(\sqrt{\Delta_0})$.  A root $\beta$ is given by
\begin{eqnarray*}
	\beta & = & \frac{-b \pm \sqrt{b^2-4c}}{2} \\
	u + v \sqrt{\Delta_0} & = & -\frac{b}{2} \pm \frac{\sqrt{b^2-4c}}{2}
\end{eqnarray*}
for $u,v \in \QQ$ and $b,c,\Delta_0 \in \ZZ$.
Equating rational and irrational parts we have
\[
	u = -\frac{b}{2}
\]
and
\begin{eqnarray*}
	v \sqrt{\Delta_0} & = & \frac{\sqrt{b^2 -4c}}{2} \\
	\sqrt{v^2 \Delta_0} & = & \sqrt{\frac{b^2}{4} - c} \\
	v^2 \Delta_0 & = & \frac{b^2}{4} - c.
\end{eqnarray*}


There are two cases: when $b$ is even, and when $b$ is odd.  First, suppose $b$ is even.  Then $b=2b'$ for $b' \in \ZZ$.  This implies that $u = -b'$ and $u \in \ZZ$.  Since $b', c, \Delta_0 \in \ZZ$ we have
\begin{eqnarray*}
	v^2 \Delta_0 & = & \frac{(2b')^2}{4} - c \\
	& = & b'^2 - c
\end{eqnarray*}
and $v \in \ZZ$.  Therefore, when $u, v \in \ZZ$ it follows that $\beta = u + v \sqrt{\Delta_0}$ is a quadratic integer.

\bigbreak
Now suppose $b$ is odd; then $b=2b' + 1$ for $b' \in \ZZ$.  Again, equating rational and irrational parts we have
\[
	u = - \frac{b}{2} = -b' - \frac{1}{2}
\]
and
\begin{eqnarray*}
	v^2 \Delta_0 & = & \frac{(2b'+1)^2}{4} - c \\
	& = & b'^2 + b' + \frac{1}{4} -c.
\end{eqnarray*}
Let $v = \frac{j}{2}$ for some $j \in \ZZ$.  
\begin{equation*}
\begin{array}{l l l l}
	& v^2 \Delta_0 & = & b'^2 + b' + \frac{1}{4} -c \\
	& \frac{j^2}{4} \Delta_0 & = & b'^2 + b' + \frac{1}{4} -c \\
	& j^2 \Delta_0 & = & 4b'^2 + 4b' + 1 - 4c \\
	\Rightarrow & j^2 \Delta_0 & \equiv & 1 \pmod 4.
\end{array}
\end{equation*}


This is only possible when $j \equiv 1,3 \pmod 4$ and $\Delta_0 \equiv 1 \pmod 4$.  Suppose $\Delta_0 \equiv 1 \pmod 4$, let $v = j/2$ and $u = i + j/2$ for some $i, j \in \ZZ$.  Then
\begin{eqnarray*}
	\beta & = & u + v \sqrt{\Delta_0} \\
	& = & i + \frac{j}{2} + \frac{j \sqrt{\Delta_0}}{2} \\
	& = & i + j \left( \frac{1 + \sqrt{\Delta_0}}{2} \right).
\end{eqnarray*}
The case of $\beta = u + v \sqrt{\Delta_0}$ when $u,v \in \ZZ$ is captured by the above when $j$ is even. When $\Delta_0 \not \equiv 1 \pmod 4$, there are no solutions for $b$ odd, so the quadratic integers are characterized exclusively by $\beta = u + v \sqrt{\Delta_0}$ for $u,v \in \ZZ$.

\bigbreak
As a result, we can completely characterize the quadratic integers with the following: Let
\[
	\omega = \begin{cases}
		\sqrt{\Delta} & \textrm{ when } \Delta \not \equiv 1 \pmod 4 \\
		(1+\sqrt{\Delta})/2 & \textrm{ when } \Delta \equiv 1 \pmod 4
	\end{cases}
\]
then $\beta = i + j \omega$ where $i,j \in \ZZ$ is an algebraic integer of $\KK = \QQ(\Delta)$.


\bigbreak
\section{Maximal Order of Algebraic Integers}
In order to define the maximal order of algebraic integers in a quadratic number field, we first begin by defining the \emph{$(\ZZ)$-module}.  Given a subset of a number field $X = \{ \xi_1, \xi_2, \xi_3, ..., \xi_n \} \subseteq \KK$, the \emph{$(\ZZ)$-module} $\MM$ is generated by $X$ as
\[
	M = \left \{ \sum_{i}^n x_i \xi_i : x_i \in \ZZ \right \}
\]
and is denoted
\[
	\MM = [ \xi_1, \xi_2, ..., \xi_n ].
\]
An \emph{order} of $\KK$ is a module $\MM$ of $\KK$ such that $1 \in \MM$ and 

\cite{JacobsonCh4}[p.~80]

The module $[1, \omega_0]$ is the module of algebraic integers in the quadratic number field $\QQ(\Delta_0)$.


\section{Imaginary Quadratic Number Fields}



\end{document}

