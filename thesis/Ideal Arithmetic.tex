\documentclass{ucalgthes1}   
\usepackage[letterpaper,top=1in, bottom=1.22in, left=1.40in, right=0.850in]{geometry}
\usepackage{fancyhdr}
\fancyhead{}
\fancyfoot{}
\renewcommand{\headrulewidth}{0pt}
\fancyhead[RO,LE]{\thepage}  
\usepackage{hyperref}

\usepackage{algorithmic}
\usepackage{amsfonts}
\usepackage{amssymb}
\usepackage{amsmath}
\usepackage{amsthm}
\usepackage{comment}

\theoremstyle{plain}
\newtheorem{thm}{Theorem}[section]
\newtheorem{lemma}[thm]{Lemma}
\newtheorem{prop}[thm]{Proposition}
\newtheorem{cor}[thm]{Corollary}
\theoremstyle{definition}
\newtheorem{defn}[thm]{Definition}


\newcommand{\CC}{\mathbb{C}}
\newcommand{\NN}{\mathbb{N}}
\newcommand{\RR}{\mathbb{R}}
\newcommand{\KK}{\mathbb{K}}
\newcommand{\MM}{\mathcal{M}}
\newcommand{\OO}{\mathcal{O}}
\newcommand{\ZZ}{\mathbb{Z}}
\newcommand{\QQ}{\mathbb{Q}}
\newcommand{\NP}{\textrm{NP}}
\newcommand{\RRgtz}{\mathbb{R}_{>0}}
\newcommand{\ZZgtz}{\mathbb{Z}_{>0}}
\newcommand{\ZZgez}{\mathbb{Z}_{\ge 0}}
\newcommand{\QQgtz}{\mathbb{Q}_{>0}}
\newcommand{\QQgez}{\mathbb{Q}_{\ge 0}}
\newcommand{\matrixot}[2]{\left( \begin{array}{r} #1 \\ #2 \end{array} \right)}
\newcommand{\matrixtt}[4]{\left( \begin{array}{rr} #1 & #2 \\ #3 & #4 \end{array} \right)}
\newcommand{\ntoinfty}{\lim_{n \rightarrow \infty}}
\newcommand{\floor}[1]{\left\lfloor #1 \right\rfloor}
\newcommand{\ceil}[1]{\left\lceil #1 \right\rceil}

%\setlength{\parindent}{0pt}
%\setlength{\parskip}{2ex} 

\begin{document}


\setcounter{chapter}{1}
\chapter{Ideal Arithmetic}

In this chapter, we present concepts and algorithms relevant to the ideal class group of imaginary quadratic number fields relevant to the work of this thesis.  The material in this chapter is found in basic number theory texts, especially (TODO).  For a more definitive treatment of the material presented here, we direct the reader towards this material.  We provide additional citations when necessary.  

%%%%%%%%%%%%%%%%%%%%%
% QUADRATIC NUMBERS %
%%%%%%%%%%%%%%%%%%%%%
\section{Quadratic Numbers}

\begin{defn}
A complex number $\alpha$ is called an \emph{algebraic number} if it is a root of a polynomial $f(x) \in \QQ[x]$.
\end{defn}
Let $f'(x) = cf(x)$ for some $c \in \ZZ$ such that $f'(x) \in \ZZ[x]$.  Then $f'(x)$ has the same roots as $f(x)$.  
\begin{defn}
If $f'(x) \in \ZZ[x]$ is a monic polynomial, then a root $\alpha$ is called an \emph{algebraic integer}. 
\end{defn}
Given an algebraic number $\alpha$, we create an algebraic number field as the extension field of $\QQ$ adjoined $\alpha$, denoted $\QQ(\alpha)$. Let $\KK$ be our algebraic number field:
\[
	\KK = \QQ(\alpha) = \left\{ \frac{f(\alpha)}{g(\alpha)} : f(x), g(x) \in \QQ[x]; g(\alpha) \ne 0 \right\}.
\]

\bigbreak
The rational numbers $\QQ$ are algebraic numbers of degree 1, i.e. $a/b \in \QQ$ is a root of $bx - a$.  Quadratic numbers are algebraic numbers of degree 2.  For example, given a polynomial $f(x) = ax^2 + bx + c$ where $f(x) \in \ZZ[x]$, the root $\alpha$ is given by the quadratic formula
\[
	\alpha = \frac{-b \pm \sqrt{b^2 - 4ac}}{2a}.
\]
If we let $\Delta = b^2 -4ac$ such that $\Delta \in \ZZ$ is the discriminant of $f(x)$, our quadratic number field is 
\[
	\KK = \QQ(\alpha) = \QQ(\sqrt{\Delta}) = \{u + v\sqrt{\Delta} : u,v \in \QQ\}.
\]
Notice that
\begin{eqnarray*}
\begin{array}{l l l l}
	\Delta & \equiv & b^2-4ac & \pmod 4 \\
	& \equiv & b^2 & \pmod 4 \\
	& \equiv & 0,1 & \pmod 4.
\end{array}
\end{eqnarray*}
Also, if $\Delta = f^2 \Delta_0$ for some $f \in \ZZ$ where $\Delta_0$ is squarefree, then $\KK = \QQ(\Delta_0)$.


\bigbreak
\section{Quadratic Integers}
Let us characterize the algebraic integers of $\QQ(\sqrt{\Delta_0})$.  An algebraic integer $\beta$ of $\QQ(\sqrt{\Delta_0})$ is $\beta = u+v \sqrt{\Delta_0}$ such that $\beta$ is the root of a monic quadratic polynomial $f(x) = x^2+bx+c \in \ZZ[x]$ for some $b,c \in \ZZ$.  The root is given by
\begin{eqnarray*}
	\beta & = & \frac{-b \pm \sqrt{b^2-4c}}{2} \\
	u + v \sqrt{\Delta_0} & = & -\frac{b}{2} \pm \frac{\sqrt{b^2-4c}}{2}
\end{eqnarray*}
for $u,v \in \QQ$ and $b,c,\Delta_0 \in \ZZ$.
Equating rational and irrational parts we have
\[
	u = -\frac{b}{2}
\]
and
\begin{eqnarray*}
	v \sqrt{\Delta_0} & = & \frac{\sqrt{b^2 -4c}}{2} \\
	\sqrt{v^2 \Delta_0} & = & \sqrt{\frac{b^2}{4} - c} \\
	v^2 \Delta_0 & = & \frac{b^2}{4} - c.
\end{eqnarray*}


There are two cases: when $b$ is even, and when $b$ is odd.  First, suppose $b$ is even.  Then $b=2b'$ for some $b' \in \ZZ$.  This implies that $u = -b'$ and $u \in \ZZ$.  Since $b', c, \Delta_0 \in \ZZ$ we have
\begin{eqnarray*}
	v^2 \Delta_0 & = & \frac{(2b')^2}{4} - c \\
	& = & b'^2 - c
\end{eqnarray*}
and $v \in \ZZ$.  Therefore, when $u, v \in \ZZ$ it follows that $\beta = u + v \sqrt{\Delta_0}$ is a quadratic integer.

\bigbreak
Now suppose $b$ is odd; then $b=2b' + 1$ for some $b' \in \ZZ$.  Again, equating rational and irrational parts we have
\[
	u = - \frac{b}{2} = -b' - \frac{1}{2}
\]
and
\begin{eqnarray*}
	v^2 \Delta_0 & = & \frac{(2b'+1)^2}{4} - c \\
	& = & b'^2 + b' + \frac{1}{4} -c.
\end{eqnarray*}
Let $v = \frac{j}{2}$ for some $j \in \ZZ$.  
\begin{equation*}
\begin{array}{l l l l}
	& v^2 \Delta_0 & = & b'^2 + b' + \frac{1}{4} -c \\
	& \frac{j^2}{4} \Delta_0 & = & b'^2 + b' + \frac{1}{4} -c \\
	& j^2 \Delta_0 & = & 4b'^2 + 4b' + 1 - 4c \\
	\Rightarrow & j^2 \Delta_0 & \equiv & 1 \pmod 4.
\end{array}
\end{equation*}

\noindent
This is only possible when $j \equiv 1,3 \pmod 4$ and $\Delta_0 \equiv 1 \pmod 4$.  Suppose \break $\Delta_0 \equiv 1 \pmod 4$, let $v = j/2$ and $u = i + j/2$ for some $i, j \in \ZZ$.  Then
\begin{eqnarray*}
	\beta & = & u + v \sqrt{\Delta_0} \\
	& = & i + \frac{j}{2} + \frac{j \sqrt{\Delta_0}}{2} \\
	& = & i + j \left( \frac{1 + \sqrt{\Delta_0}}{2} \right).
\end{eqnarray*}
The case of $\beta = u + v \sqrt{\Delta_0}$ when $u,v \in \ZZ$ is captured above when $j$ is even. When $\Delta_0 \not \equiv 1 \pmod 4$, there are no solutions for $b$ odd.  We summarize this with the following definition.

\begin{defn}
\label{defn:quadraticInteger}
A \emph{quadratic integer} $\beta$ is an algebraic integer of $\QQ(\sqrt{\Delta_0})$ such that $\beta = i + j \omega_0$ for some $i,j \in \ZZ$ and
\[
	\omega_0 = \begin{cases}
		\sqrt{\Delta_0} & \textrm{ when } \Delta_0 \not \equiv 1 \pmod 4 \\
		\frac{1+\sqrt{\Delta_0}}{2} & \textrm{ when } \Delta_0 \equiv 1 \pmod 4.
	\end{cases}
\]
\end{defn}
\bigbreak


%%%%%%%%%%%%%%%%%
% MAXIMAL ORDER %
%%%%%%%%%%%%%%%%%
\bigbreak
\section{Maximal Order of Algebraic Integers}
The set of all algebraic integers of a number field $\KK$ is a \emph{maximal order} of $\KK$.  To state a \emph{maximal order} of $\KK$, we begin by defining a \emph{$\ZZ$-module}.  

\begin{defn}
Given a subset of a number field $X = \{ \xi_1, \xi_2, \xi_3, ..., \xi_n \} \subseteq \KK$, the \emph{$\ZZ$-module} $\MM$ is generated as a linear combination of the elements of $X$:
\[
	\MM = \left \{ \sum_{i}^n x_i \xi_i : x_i \in \ZZ, \xi_i \in X \right \}.
\]
We denote this
\[
	\MM = [ \xi_1, \xi_2, ..., \xi_n ].
\]
\end{defn}

\begin{defn}
An \emph{order} $\OO$ of $\KK$ is a $\ZZ$-module of $\KK$ such that $\OO$ is a subring of $\KK$ containing 1.  A \emph{quadratic order} is a $\ZZ$-module of $\QQ(\sqrt{\Delta_0})$ such that $\OO_\Delta = [1, f\omega_0]$ where $\Delta = f^2\Delta_0$ for some $f \in \ZZ$.  We call $f$ the \emph{conductor} of $\OO_\Delta$.
\end{defn}

\bigbreak
Given $\omega_0$ from definition \ref{defn:quadraticInteger}, the module $[1, \omega_0]$ is the module of algebraic integers in the quadratic number field $\KK=\QQ(\sqrt{\Delta_0})$.  We denote this as $\OO_{\KK}$.  The order $\OO_\KK$ is of maximal size since any order $\OO_\Delta = [1,f\omega_0]$ is a subring of $\OO_\KK$. In general, we express the order $\OO_\Delta$ as the $\ZZ$-module:
\[
	[1,f\omega_0] = \left[1, \frac{\Delta + \sqrt{\Delta}}{2} \right].
\]
When $\Delta$ is assumed to be known, we write $\OO$ for $\OO_\Delta$ and $\omega$ for $f\omega_0$.


%%%%%%%%%%
% IDEALS %
%%%%%%%%%%
\bigbreak
\section{Ideals of $\OO_\Delta$}

Here we define concepts of ideals of a quadratic order $\OO_\Delta$.

\begin{defn}
An integral \emph{ideal} $\mathfrak{a}$ of an order $\OO$, known as an \emph{$\OO$-ideal}, is an additive subgroup of $\OO$ such that $\xi \mathfrak{a} \subseteq \mathfrak{a}$ for any $\xi \in \OO$.
\end{defn}

We denote an ideal $\mathfrak{a}$ as a two-dimensional $\OO$-module
\[
	\alpha \OO + \beta \OO = (\alpha, \beta)
\]
for $\alpha, \beta \in \OO$.  Computationally, this is represented as a two-dimensional $\ZZ$-module
\begin{equation}
\label{eq:idealZModule}
	(\alpha, \beta) = \left[sa, s\left( \frac{b+ \sqrt{\Delta}}{2} \right) \right] = s \left(a \ZZ + \frac{b + \sqrt{\Delta}}{2} \ZZ \right)
\end{equation}
for $a,b,s,\Delta \in \ZZ$ where $s$ is the largest integral term dividing both generators of the $\ZZ$-module. Furthermore, $s$, $a$, and $b$ are unique, and $b$ is unique $\bmod~2a$ since
\begin{equation*}
	\left[sa, s\left(\frac{2a+b+\sqrt{\Delta}}{2}\right)\right] = \left[s(a+1), s\left(\frac{b+\sqrt{\Delta}}{2}\right)\right].
\end{equation*}

\vspace{1ex}
\begin{defn}
The \emph{conjugate} of an ideal $\mathfrak{a} = (\alpha, \beta)$ is $\overline{\mathfrak{a}} = (\alpha, -\beta)$.
\end{defn}

We call an ideal $\mathfrak{a} = [sa, s(b+\sqrt{\Delta})/2]$ with $s=1$ a \emph{primitive ideal}.  An ideal $\mathfrak{a}$ is \emph{principal} if it is generated by a single element $\alpha \in \OO$. In this case, we write $\mathfrak{a} = \alpha \OO = (\alpha)$.  An $\OO$-ideal $\mathfrak{a}$ is \emph{invertible} if there exists an $\OO$-ideal $\mathfrak{b}$ such that $\mathfrak{a}\mathfrak{b}$ is principal.  When $\OO$ is maximal, all ideals of $\OO$ are invertible.  If $\OO_\Delta$ for $\Delta = f^2\Delta_0$ is non-maximal, then an $\OO$-ideal $\mathfrak{a}$ is invertible if $\gcd(N(\mathfrak{a}),f)=1$ where $N(\mathfrak{a})$ is the \emph{norm} defined as the number of distinct cosets in the group $\OO$ modulo $\mathfrak{a}$, denoted $|\OO/\mathfrak{a}|$.  For an invertible ideal $\mathfrak{a} = [sa, s(b+\sqrt{\Delta})/2]$, the norm is $N(\mathfrak{a}) = s^2a$ \cite{Jac99}.  When two $\OO$-ideals $\mathfrak{a}$ and $\mathfrak{b}$ are invertible, the norm is distributive, $N(\mathfrak{a}\mathfrak{b}) = N(\mathfrak{a})N(\mathfrak{b})$.
\end{defn}



Two ideals $\mathfrak{a}$ and $\mathfrak{b}$ of $\OO_$ are \emph{equivalent} if there exist principal ideals $(\alpha)$ and $(\beta)$ with  $\alpha, \beta \in \OO$ and $\alpha\beta \neq 0$ such that $(\alpha)\mathfrak{a} = (\beta)\mathfrak{b}$.  Given an $\OO$-ideal $\mathfrak{a}$, the set of all $\OO$-ideals which are equivalent to $\mathfrak{a}$ is donated $[\mathfrak{a}]$ and is called an \emph{ideal class} of $\OO$.  This brings us to the ideal class group.


%%%%%%%%%%%%%%%%%%%%%
% IDEAL CLASS GROUP %
%%%%%%%%%%%%%%%%%%%%%
\bigbreak
\section{Ideal Class Group}

Using the above notions of ideals in $\OO_\Delta$, we define the \emph{ideal class group} $Cl_\Delta$ as the set of all equivalence classes of invertible $\OO$-ideals. The group operator is multiplication with reduction on ideal classes.  In section \ref{section:computerRepresentation} we give a concrete representation of the ideal class group and the operations on elements of this group.

\begin{defn}
Ideals classes are represented in reduced form. Let $\mathfrak{a}$ be an $\OO$-ideal. Then $\mathfrak{a}$ is said to be \emph{reduced} if $\mathfrak{a}$ is primitive, and there does not exist a non-zero element $\beta \in \mathfrak{a}$ such that $|\beta| < N(\mathfrak{a})$ and $|\overline{\beta}| < N(\mathfrak{a})$.  In otherwords, $N(\mathfrak{a})$ is a minimum in $\mathfrak{a}$ \cite{Jac09}.  See subsection \ref{subsection:idealClassReduction} for the reduction algorithm used.
\end{defn}


Multiplication of ideal classes is as follows: given reduced representatives $\mathfrak{a}$ and $\mathfrak{b}$ of ideal classes as $\ZZ$-modules, let
\begin{eqnarray*}
	\mathfrak{a} & = & a_1 \ZZ + \frac{b_1 + \sqrt{\Delta}}{2} \ZZ \\
	\mathfrak{b} & = & a_2 \ZZ + \frac{b_2 + \sqrt{\Delta}}{2} \ZZ.
\end{eqnarray*}

There is no $s$ term in the above, as reduced representatives are primitive.  By definition we have
\begin{equation}
\begin{split}
	\mathfrak{a} \mathfrak{b} & = a_1a_2 \ZZ + a_1 \frac{b_2 + \sqrt{\Delta}}{2} \ZZ + a_2 \frac{b_1 + \sqrt{\Delta}}{2} \ZZ + \frac{b_1 + \sqrt{\Delta}}{2} \cdot \frac{b_2 + \sqrt{\Delta}}{2} \ZZ \\
	& = a_1a_2 \ZZ + \frac{a_1b_2 + a_1\sqrt{\Delta}}{2} \ZZ + \frac{a_2b_1 + a_2\sqrt{\Delta}}{2} \ZZ + \frac{b_1b_2 + (b_1+b_2)\sqrt{\Delta} + \Delta}{4} \ZZ \label{eq:composeExpanded}
\end{split}
\end{equation}

\noindent
However, we want to reformulate this given the representation in equation \eqref{eq:idealZModule} as
\[
	\mathfrak{a} \mathfrak{b} = sa \ZZ + s \left(\frac{b + \sqrt{\Delta}}{2}\right) \ZZ
\]
for some $s, a, b \in \ZZ$.  Since the norm is distributive we have
\begin{eqnarray*}
	&& s^2a = N(\mathfrak{a}\mathfrak{b}) = N(\mathfrak{a})N(\mathfrak{b}) = a_1 a_2 \\
	& \Rightarrow & a = \frac{a_1a_2}{s}.
\end{eqnarray*}
Now, by the second term of equation \eqref{eq:composeExpanded} we know that $(a_1b_2 + a_1\sqrt{\Delta})/2 \in \mathfrak{a}\mathfrak{b}$.  It follows that there is some $x,y \in \ZZ$ such that
\[
	\frac{a_1b_2 + a_1\sqrt{\Delta}}{2} = xsa + ys\left(\frac{b+\sqrt{\Delta}}{2}\right).
\]
Equating irrational parts we have:
\begin{eqnarray*}
	&& \frac{a_1\sqrt{\Delta}}{2} = \frac{ys\sqrt{\Delta}}{2} \\
	& \Rightarrow & a_1 = ys \\
	& \Rightarrow & s~|~a_1.
\end{eqnarray*}
\noindent
Similarly, by the third and fourth terms of equation \eqref{eq:composeExpanded} we have $(a_2b_1+a_2\sqrt{\Delta})/2 \in \mathfrak{a}\mathfrak{b}$ which implies that $s~|~a_2$ and $(b_1b_2 + (b_1+b_2)\sqrt{\Delta} + \Delta)/4 \in \mathfrak{a}\mathfrak{b}$ which implies that \break $s~|~(b_1+b_2)/2$. 

By the second generator $(sb+s\sqrt\Delta)/2$ of $\mathfrak{a}\mathfrak{b}$ there exists $m, n, p, q \in \ZZ$ such that
\begin{equation}
\begin{split}
	\frac{sb+s\sqrt\Delta}{2} & = ma_1a_2 + n\frac{a_1b_2+a_1\sqrt\Delta}{2} + p\frac{a_2b_1 + a_2\sqrt{\Delta}}{2} + q\frac{b_1b_2 + (b_1+b_2)\sqrt{\Delta} + \Delta}{4} \\
	& = ma_1a_2 + n\frac{a_1b_2}{2} + p\frac{a_2b_1}{2} + q\frac{b_1b_2 + \Delta}{4} + \left(n\frac{a_1}{2} + p\frac{a_2}{2} + q\frac{b_1+b_2}{4}\right)\sqrt\Delta. \label{eq:composeSecond}
\end{split}
\end{equation}
Again, by equating irrational parts we have:
\begin{eqnarray*}
	&& \frac{s\sqrt\Delta}{2} = \left(n\frac{a_1}{2} + p\frac{a_2}{2} + q\frac{b_1+b_2}{4}\right)\sqrt\Delta \\
	& \Rightarrow & s = na_1 + pa_2 + q\frac{b_1+b_2}{2}.
\end{eqnarray*}
Since $s~|~\gcd(a_1, a_2, (b_1+b_2)/2)$, we have that $s = \gcd(a_1, a_2, (b_1+b_2)/2)$.  

It remains to compute $b \pmod{2a}$.  Recall that $a = (a_1a_2)/s$.  This time, by equating the rational parts of \eqref{eq:composeSecond} we have:
\begin{eqnarray*}
	\frac{sb}{2} & = & ma_1a_2 + n\frac{a_1b_2}{2} + p\frac{a_2b_1}{2} + q\frac{b_1b_2 + \Delta}{4} \\
	\Rightarrow b & = & \frac{2ma_1a_2}{s} + n\frac{a_1b_2}{s} + p\frac{a_2b_1}{s} + q\frac{b_1b_2 + \Delta}{2s} \\
	\Rightarrow b & \equiv & n\frac{a_1b_2}{s} + p\frac{a_2b_1}{s} + q\frac{b_1b_2 + \Delta}{2s} \pmod{2a}
\end{eqnarray*}


%%%%%%%%%%%%%%%%
% TO USE STILL %
%%%%%%%%%%%%%%%%
\bigbreak
\section{To use still}

The number of elements in the class group is denoted $h_\Delta = |Cl_\Delta|$ and is known as the \emph{class number}.


\begin{defn}
A \emph{prime ideal} of $\OO$ is an invertible $\OO$-ideal $\mathfrak{p} \neq \OO$ with the property that if $\mathfrak{p} | \mathfrak{a}\mathfrak{b}$, for $\OO$-ideals $\mathfrak{a}$ and $\mathfrak{b}$, then $\mathfrak{p} | \mathfrak{a}$ or $\mathfrak{p} | \mathfrak{b}$ \cite{JacobsonCh4}[p.~93].
\end{defn}
In the ideal class group, the reduced form of a prime ideal $\mathfrak{p}$ is given by $[p, \beta]$ where $p \in \ZZ$ is prime.


%%%%%%%%%%%%%%%%%%%%%%%%%%%
% COMPUTER REPRESENTATION %
%%%%%%%%%%%%%%%%%%%%%%%%%%%
\bigbreak
\section{Computer Representation}
\label{section:computerRepresentation}

\subsection{Ideal Class Reduction}
\label{subsection:idealClassReduction}

\end{document}

