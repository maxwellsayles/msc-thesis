\documentclass[11pt, letterpaper]{book}

\usepackage{algorithmic}
\usepackage{amsfonts}
\usepackage{amssymb}
\usepackage{amsmath}
\usepackage{amsthm}
\usepackage{fullpage}
\usepackage{comment}

\newtheorem*{thm}{Theorem}
\newtheorem*{lem}{Lemma}
\newtheorem*{cor}{Corollary}
\theoremstyle{definition}
\newtheorem*{defn}{Definition}


\parindent 0ex

\newcommand{\CC}{\mathbb{C}}
\newcommand{\NN}{\mathbb{N}}
\newcommand{\RR}{\mathbb{R}}
\newcommand{\RRgtz}{\mathbb{R}_{>0}}
\newcommand{\ZZ}{\mathbb{Z}}
\newcommand{\ZZgtz}{\mathbb{Z}_{>0}}
\newcommand{\ZZgez}{\mathbb{Z}_{\ge 0}}
\newcommand{\QQ}{\mathbb{Q}}
\newcommand{\QQgtz}{\mathbb{Q}_{>0}}
\newcommand{\QQgez}{\mathbb{Q}_{\ge 0}}
\newcommand{\NP}{\textrm{NP}}
\newcommand{\matrixot}[2]{\left( \begin{array}{r} #1 \\ #2 \end{array} \right)}
\newcommand{\matrixtt}[4]{\left( \begin{array}{rr} #1 & #2 \\ #3 & #4 \end{array} \right)}
\newcommand{\ntoinfty}{\lim_{n \rightarrow \infty}}
\newcommand{\floor}[1]{\left\lfloor #1 \right\rfloor}
\newcommand{\ceil}[1]{\left\lceil #1 \right\rceil}


\begin{document}

\setcounter{chapter}{1}
\chapter{Ideal Arithmetic}

An \emph{algebraic number} $\alpha$ is a complex number that is a root of some unique irreducible (over $\QQ$) monic polynomial $f(x) \in \QQ[x]$.  An algebraic number field is any subfield of the set of all algebraic numbers.  Given an algebraic number $\alpha$ we can create an algebraic number field as the extension field of $\QQ$ adjoined $\alpha$ (denoted $\QQ(\alpha)$). Let
\[
	\mathbb{K} = \QQ(\alpha) = \left\{ \frac{f(\alpha)}{g(\alpha)} : f(x), g(x) \in \QQ[x]; g(\alpha) \ne 0 \right\}
\]
Then $\mathbb{k}$ is our algebraic number field.

\bigbreak
An \emph{algebraic integer} is a complex number that is a root of a monic polynomial $g(x) \in \ZZ[x]$.

\bigbreak
Rational numbers are algebraic numbers of degree 1, i.e. $a/b \in \QQ$ is the root of $bx - a$.

\bigbreak
Quadratic numbers are algebraic numbers of degree 2.  Given a degree 2 polynomial $ax^2 + bx + c$ the root is given by the quadratic formula
\[
	\frac{-b \pm \sqrt{b^2 - 4ac}}{2a}
\]
If we let $D = b^2 -4ac$

\section{Imaginary Quadratic Number Fields}



\end{document}

