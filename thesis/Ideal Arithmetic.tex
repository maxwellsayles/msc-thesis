\documentclass[11pt, letterpaper]{book}

\usepackage{algorithmic}
\usepackage{amsfonts}
\usepackage{amssymb}
\usepackage{amsmath}
\usepackage{amsthm}
\usepackage{fullpage}
\usepackage{comment}

\newtheorem*{thm}{Theorem}
\newtheorem*{lem}{Lemma}
\newtheorem*{cor}{Corollary}
\theoremstyle{definition}
\newtheorem*{defn}{Definition}


\parindent 0ex

\newcommand{\CC}{\mathbb{C}}
\newcommand{\NN}{\mathbb{N}}
\newcommand{\RR}{\mathbb{R}}
\newcommand{\KK}{\mathbb{K}}
\newcommand{\MM}{\mathcal{M}}
\newcommand{\OO}{\mathcal{O}}
\newcommand{\RRgtz}{\mathbb{R}_{>0}}
\newcommand{\ZZ}{\mathbb{Z}}
\newcommand{\ZZgtz}{\mathbb{Z}_{>0}}
\newcommand{\ZZgez}{\mathbb{Z}_{\ge 0}}
\newcommand{\QQ}{\mathbb{Q}}
\newcommand{\QQgtz}{\mathbb{Q}_{>0}}
\newcommand{\QQgez}{\mathbb{Q}_{\ge 0}}
\newcommand{\NP}{\textrm{NP}}
\newcommand{\matrixot}[2]{\left( \begin{array}{r} #1 \\ #2 \end{array} \right)}
\newcommand{\matrixtt}[4]{\left( \begin{array}{rr} #1 & #2 \\ #3 & #4 \end{array} \right)}
\newcommand{\ntoinfty}{\lim_{n \rightarrow \infty}}
\newcommand{\floor}[1]{\left\lfloor #1 \right\rfloor}
\newcommand{\ceil}[1]{\left\lceil #1 \right\rceil}


\begin{document}


\setcounter{chapter}{1}
\chapter{Ideal Arithmetic}


The material in this section is covered in depth in TODO.  Here we provide as a reminder some number theoretic concepts relevant to ideal arithmetic as it applies to this thesis.


\bigbreak
A complex number $\alpha$ is called an \emph{algebraic number} if it is a root of a polynomial $f(x) \in \QQ[x]$.  The polynomial $f(x) \in \QQ[x]$ can be expressed as a unique irreducible (over $\QQ$) monic polynomial.  If $\alpha$ is a root of a monic polynomial $f(x) \in \ZZ[x]$, then it is an \emph{algebraic integer}.  Given an algebraic number $\alpha$, we can create an algebraic number field as the extension field of $\QQ$ adjoined $\alpha$, denoted $\QQ(\alpha)$. Let $\KK$ be our algebraic number field; then
\[
	\KK = \QQ(\alpha) = \left\{ \frac{f(\alpha)}{g(\alpha)} : f(x), g(x) \in \QQ[x]; g(\alpha) \ne 0 \right\}.
\]


\bigbreak
The rational numbers $\QQ$ are algebraic numbers of degree 1, i.e. $a/b \in \QQ$ is a root of $x - a/b$.  Quadratic numbers are algebraic numbers of degree 2.  For example, given a polynomial $f(x) = ax^2 + bx + c$ where $f(x) \in \QQ[x]$, the root is given by the quadratic formula
\[
	\alpha = \frac{-b \pm \sqrt{b^2 - 4ac}}{2a}.
\]
If we let $\Delta = b^2 -4ac$ be the discriminant of $f(x)$, our quadratic number field is $\KK = \QQ(\alpha) = \QQ(\sqrt{\Delta})$.  If $\Delta = f^2 \Delta_0$ where $\Delta_0$ is squarefree, then $\KK = \QQ(\Delta_0)$.

\bigbreak
TODO: why?
Let
\[
	r = \begin{cases}
		1 \textrm{ when } \Delta_0 \not \equiv 1 \pmod 4 \\
		2 \textrm{ when } \Delta_0 \equiv 1 \pmod 4
	\end{cases}
\]
and $\omega_0 = (r-1+\sqrt{\Delta_0})/r$.  Then $\beta$ is an algebraic integer of $\KK = \QQ(\Delta_0)$ if and only if $\beta = x + y \omega_0$ for $x,y \in \ZZ$ $\cite{JacobsonCh4}[p.~77]$.

\bigbreak
Let $A$ be any additive abelian group, and $\MM$ is an additive abelian subgroup of $A$, then $\MM$ is a \emph{$(\ZZ)$-module} of $A$.



\section{Imaginary Quadratic Number Fields}



\end{document}

