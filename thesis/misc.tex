\documentclass[11pt, letterpaper]{article}

\usepackage{algorithmic}
\usepackage{amsfonts}
\usepackage{amssymb}
\usepackage{amsmath}
\usepackage{amsthm}
\usepackage{fullpage}
\usepackage{comment}

\newtheorem*{thm}{Theorem}
\newtheorem*{lem}{Lemma}
\newtheorem*{cor}{Corollary}
\theoremstyle{definition}
\newtheorem*{defn}{Definition}


\parindent 0ex

\newcommand{\CC}{\mathbb{C}}
\newcommand{\NN}{\mathbb{N}}
\newcommand{\RR}{\mathbb{R}}
\newcommand{\RRgtz}{\mathbb{R}_{>0}}
\newcommand{\ZZ}{\mathbb{Z}}
\newcommand{\ZZgtz}{\mathbb{Z}_{>0}}
\newcommand{\ZZgez}{\mathbb{Z}_{\ge 0}}
\newcommand{\QQ}{\mathbb{Q}}
\newcommand{\QQgtz}{\mathbb{Q}_{>0}}
\newcommand{\QQgez}{\mathbb{Q}_{\ge 0}}
\newcommand{\NP}{\textrm{NP}}
\newcommand{\matrixot}[2]{\left( \begin{array}{r} #1 \\ #2 \end{array} \right)}
\newcommand{\matrixtt}[4]{\left( \begin{array}{rr} #1 & #2 \\ #3 & #4 \end{array} \right)}
\newcommand{\ntoinfty}{\lim_{n \rightarrow \infty}}
\newcommand{\floor}[1]{\left\lfloor #1 \right\rfloor}
\newcommand{\ceil}[1]{\left\lceil #1 \right\rceil}


\begin{document}

\section{Improvements}
Instead of using a sieve to flag which numbers are coprime and which are not within the area of interest, perform another pass over the sieve and build a list of gaps from one coprime to another.  This is smaller, and is constant time to jump from one coprime to another.

\section{Description}
The algorithm works by taking some prime form $f$ in the class group of $-N$, exponentiating it by the odd part of some large power primorial $P$.  Let $b=f^P$.  We then exponentiate $b$ by some sufficiently large $2^i$ and then perform a primorial steps search for the order of this element.  Once we know that, we know a large multiple of the order of $f$, more specifically, we know the odd order of the element $b$.  We exponentiate $b$ by the odd order and then square until we find an ambiguous form, i.e. the square root of the identity.

\bigbreak
This algorithm may fail to factor given a prime form $f$ and a discriminant $-N$ when one of several things happen:
\begin{itemize}
\item The number of steps for the primorial steps search is bounded and we do not find the order within the given number of steps.  In this case, the order of the element $f$ is not sufficiently smooth.  

\item We find the order of the prime form $f$, but it is a multiple of $N = p /cdot q$ and so does not lead to a non-trivial factorization of $N$.
\end{itemize}

at this point, we can either switch to the next prime form, or change the multiplier of $N$. By changing the primeform, we hope that the new form has smooth order and it not a multiple of $N$.  If we are unlucky and the class group of $-N$ has a large prime in its class number, then by switching multipliers to some other class group $-kN$, we hope to find a class group whose class number is smooth and hence has prime forms with smooth order.

\bigbreak
Investigate if there are any particular multipliers that we would want to use.




\end{document}

