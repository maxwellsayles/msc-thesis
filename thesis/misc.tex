\documentclass[11pt, letterpaper]{article}

\usepackage{algorithmic}
\usepackage{amsfonts}
\usepackage{amssymb}
\usepackage{amsmath}
\usepackage{amsthm}
\usepackage{fullpage}
\usepackage{comment}

\newtheorem*{thm}{Theorem}
\newtheorem*{lem}{Lemma}
\newtheorem*{cor}{Corollary}
\theoremstyle{definition}
\newtheorem*{defn}{Definition}


\parindent 0ex

\newcommand{\CC}{\mathbb{C}}
\newcommand{\NN}{\mathbb{N}}
\newcommand{\RR}{\mathbb{R}}
\newcommand{\RRgtz}{\mathbb{R}_{>0}}
\newcommand{\ZZ}{\mathbb{Z}}
\newcommand{\ZZgtz}{\mathbb{Z}_{>0}}
\newcommand{\ZZgez}{\mathbb{Z}_{\ge 0}}
\newcommand{\QQ}{\mathbb{Q}}
\newcommand{\QQgtz}{\mathbb{Q}_{>0}}
\newcommand{\QQgez}{\mathbb{Q}_{\ge 0}}
\newcommand{\NP}{\textrm{NP}}
\newcommand{\matrixot}[2]{\left( \begin{array}{r} #1 \\ #2 \end{array} \right)}
\newcommand{\matrixtt}[4]{\left( \begin{array}{rr} #1 & #2 \\ #3 & #4 \end{array} \right)}
\newcommand{\ntoinfty}{\lim_{n \rightarrow \infty}}
\newcommand{\floor}[1]{\left\lfloor #1 \right\rfloor}
\newcommand{\ceil}[1]{\left\lceil #1 \right\rceil}


\begin{document}

\section{Improvements}
Instead of using a sieve to flag which numbers are coprime and which are not within the area of interest, perform another pass over the sieve and build a list of gaps from one coprime to another.  This is smaller, and is constant time to jump from one coprime to another.

\section{Description}
The algorithm works by taking some prime form $f$ in the class group of $-N$, exponentiating it by the odd part of some large power primorial $P$.  Let $b=f^P$.  We then exponentiate $b$ by some sufficiently large $2^i$ and then perform a primorial steps search for the order of this element.  Once we know that, we know a large multiple of the order of $f$, more specifically, we know the odd order of the element $b$.  We exponentiate $b$ by the odd order and then square until we find an ambiguous form, i.e. the square root of the identity.

\bigbreak
This algorithm may fail to factor given a prime form $f$ and a discriminant $-N$ when one of several things happen:
\begin{itemize}
\item The number of steps for the primorial steps search is bounded and we do not find the order within the given number of steps.  In this case, the order of the element $f$ is not sufficiently smooth.  

\item We find the order of the prime form $f$, but it is a multiple of $N = p /cdot q$ and so does not lead to a non-trivial factorization of $N$.
\end{itemize}

at this point, we can either switch to the next prime form, or change the multiplier of $N$. By changing the primeform, we hope that the new form has smooth order and it not a multiple of $N$.  If we are unlucky and the class group of $-N$ has a large prime in its class number, then by switching multipliers to some other class group $-kN$, we hope to find a class group whose class number is smooth and hence has prime forms with smooth order.

\bigbreak
Investigate if there are any particular multipliers that we would want to use.



\subsection{SuperSPAR Complexity}

% TODO: Complete this section.

%This space intentionally left empty for now.

Lenstra and Pomerance \cite[Theorem 11.1]{Lenstra1992} show that for all $x$ there exist $n \le x$ such that for every negative discriminant $\Delta \equiv 0 \pmod n$, the class number $h_\Delta$ has a prime factor exceeding $x^{1/9}$.  

By \cite[Lemma 6.4]{Sutherland2007}, for the largest prime $p_t \le N^{1/2r}$ on the order, the ratio of the size of the giant steps to the number of baby steps is
\[
P_w / \phi(P_w) \sim e^\gamma \log \log N^{1/2r}
\]
where $\gamma \approx 0.5772$ is Euler's constant and $e \approx 2.71828$. This simplifies to
\begin{align*}
	e^\gamma \log \log N^{1/2r} &= \log (\log(N)/2r) \\
	&= \log \log N - \log(2r) \\
	&= \log \log N - \log r - \log2 \\
	&= \log \log N - \log(\sqrt{\log N / \log \log N}) - \log2 \\
	&= \log \log N - \log(\log N / \log\log N) / 2 - \log2 \\
	&= \log \log N - (\log\log N - \log\log\log N) / 2 - \log2 \\
	&= (\log \log N + \log\log\log N) / 2 - \log2 \\
	&\in O(\log \log N)
\end{align*}

Note that $m\phi(P_w) \approx N^{1/2r}$.  So the largest exponent discovered by the giant steps is
\[
	m^2P_w\phi(P_w) = N^{1/r} \log \log N \approx N^{1/r}.
\]

So we find the order of $\aclass$ if $\ord(\aclass)$ is $N^{1/r}$ smooth and this occurs with probability $(r/2)^{(r/2)}$.  So the expected running time is
\[
	(r/2)^{(r/2)} N^{1/r}.
\]

Substituting $r' = r/2$ in
\[
	r'^{r'} N^{1/2r'}
\]
which is minimized for $r' = \sqrt{\ln N / \ln \ln N}$ and the expected running time for $r = 2\sqrt{\ln N / \ln \ln N}$ is
\[
	\exp(\sqrt{\ln N \ln \ln N}).
\]

On page 81 of Drew's thesis he states that if $h=\ord(\alpha)$ is uniformly distributed over some appropriate range, the probability that $h$ is $h^{1/u}$-smooth is $u^{-u+o(1)}$.

So if we treat the order of a random element from the ideal class group as uniformly distributed over an appropriate range then I understand the analysis of the original SPAR as follows.

For some $r$, let $p_t$ be the largest prime $\le N^{1/2r}$. We spend $O(p_t)$ group operations on each multiplier with a probability of factoring being $\ge r^{-r}$ and so in expectation we use $r^r$ multipliers for an expected running time of $r^r N^{1/r}$.  This is minimized for $r = \sqrt{\ln N / \ln \ln N}$.

I do not understand how the value of $r$ was determined.

Now, to spend $O(p_t)$ operations on each multiplier, SPAR chooses the first $t$ primes $p_1, ..., p_t$, and computes $e_i = \max \{ v : {p_i}^v \le {p_t}^2 \}$.  Let $E = \prod {p_i}^{e_i}$.  We then exponentiate a random form $a$ to $E$.  Using binary exponentiation, this takes $O(\log E)$ operations, which is $O(p_t)$ since there are about $p_t / \log p_t$ operands in the product $E$, each of which is $2 \log p_t$ in size.  If this fails to find an ambiguous form, we do a random walk of $O(p_t)$ operations on the new form.

One thing I'm unsure about is whether $\ord(a^E)$ is guaranteed to not have any primes $p_i \le p_t$ as divisors, since SPAR uses $e_i$ bound by ${p_t}^2$.  This does not seem like a high enough bound (maybe it is asymptotically?).


Looking at Andrew's original SuperSPAR implementation, this is a little different.  We pick some $u$ (this was $r$ in SPAR).  Then compute all the primes up to $N^{1/u}$ and exponents $e_i = \max \{ v : {p_i}^v \le p_t \}$.  Here we use $p_t$ instead of ${p_t}^2$ but I don't think that matters too much.  Again, let $E$ be the product of all prime powers.

For the purpose of discussing SuperSPAR, we treat the distribution on the order of random ideals as uniform.  Let $N$ be the odd integer we wish to factor.  For some square free integer $k$, choose a discriminant $\Delta = -kN$ or $\Delta = -4kN$ such that $\Delta \equiv 0, 1 \pmod 4$, and a random prime ideal class representative $\mathfrak a$ such that $\aclass \in Cl_\Delta$.  The idea is that for some real value $u$, we will perform $O(|\Delta|^{1/u})$ group operations on $\aclass$ in an attempt to find an ambiguous idea.  Treating the order $h = \ord(\aclass)$ as uniformly distributed, we can expect $h$ to be $|\Delta|^{1/u}$-smooth with probability $u^{-u}$.  As such, we can expect to try $u^u$ different discriminants. TODO: more rigour here.

Sutherland notes \cite[p.84]{Sutherland2007} that the distribution of $\ord(\alpha_i)$ may not be uniform.

Ideally, we want to minimize the expected number of group operations that lead to a factorization of the integer $N$.  The parameters involved are the number of prime ideal classes $\aclass$ tried before switching multipliers and class groups, the bound on the largest prime in the exponent $E$, a bound on the exponents $e_i$ of the prime powers ${p_i}^{e_i}$ whose product is the exponents $E$, and a bound on the largest baby step $s$.


Cohen \cite[p.295]{Cohen1993} states a bound on the number of elements in the class group $Cl_\Delta$ as
\begin{equation}
\label{eq:hDelta}
	h_\Delta < \frac{1}{\pi} \sqrt{|\Delta|}\log{|\Delta|} \textrm{ when } \Delta < -4.
\end{equation}
Since the prime exponents $e_i$ in Stage 1 are bound by $\floor{\log_{p_i} {p_t}^2}$ instead of $\floor{\log_{p_i} h_\Delta}$, there is no guaranteed that $h' = \ord(\cclass)$ does not have some $p_i \le p_t$ as a divisor.


%If $\ord(\aclass)$ divides $\prod_{i=2}^t {p_i}^{e_i}$ then by \cite[p.292]{Schnorr1984} the probability that $\ord(\aclass)$ is even is $\ge 1/2$. 

%\subsection{How Do We Do It (Empirically)}
%\subsection{Analysis for Generic Groups}
%\subsection{Analysis for Class Group of Imaginary Quadratic Number Fields}
%\subsection{Comparison with Original SPAR}


\end{document}

