\documentclass[11pt, letterpaper]{article}

\usepackage{algorithmic}
\usepackage{amsfonts}
\usepackage{amssymb}
\usepackage{amsmath}
\usepackage{amsthm}
\usepackage{fullpage}
\usepackage{comment}

\newtheorem*{thm}{Theorem}
\newtheorem*{lem}{Lemma}
\newtheorem*{cor}{Corollary}
\theoremstyle{definition}
\newtheorem*{defn}{Definition}


\parindent 0ex

\newcommand{\CC}{\mathbb{C}}
\newcommand{\NN}{\mathbb{N}}
\newcommand{\RR}{\mathbb{R}}
\newcommand{\RRgtz}{\mathbb{R}_{>0}}
\newcommand{\ZZ}{\mathbb{Z}}
\newcommand{\ZZgtz}{\mathbb{Z}_{>0}}
\newcommand{\ZZgez}{\mathbb{Z}_{\ge 0}}
\newcommand{\QQ}{\mathbb{Q}}
\newcommand{\QQgtz}{\mathbb{Q}_{>0}}
\newcommand{\QQgez}{\mathbb{Q}_{\ge 0}}
\newcommand{\NP}{\textrm{NP}}
\newcommand{\matrixot}[2]{\left( \begin{array}{r} #1 \\ #2 \end{array} \right)}
\newcommand{\matrixtt}[4]{\left( \begin{array}{rr} #1 & #2 \\ #3 & #4 \end{array} \right)}
\newcommand{\ntoinfty}{\lim_{n \rightarrow \infty}}
\newcommand{\floor}[1]{\left\lfloor #1 \right\rfloor}
\newcommand{\ceil}[1]{\left\lceil #1 \right\rceil}


\begin{document}


\section{Motivation}
\begin{itemize}
\item Factor residues in the double small prime variant of the number field sieve
\item Other reasons why factoring small integers quickly is important
\end{itemize}

\bigbreak
\section{SPAR}
\subsection{Quadratic Forms}
\subsubsection{Ambiguous Forms}
\subsection{Multipliers}


\bigbreak
\section{Primorial Steps Algorithm}
\subsection{Baby-Step Giant-Step}
\subsection{Sub-exponential run-time}


\bigbreak
\section{Double-Base Number System}
\subsection{Chains vs Representations}
\subsection{Encoding of chains/representations}
\subsection{Sub-linearity in number of terms}

\bigbreak
\section{SuperSPAR}
\begin{itemize}
\item Putting it all together
\end{itemize}

\bigbreak
\section{Experiments}

\subsection{Coprime Finding}
\begin{itemize}
\item Wheeling
\item Sieving
\item Distance Table
\end{itemize}

\subsection{64bit, 128bit, MPZ variants}

\subsection{Extended GCD}
\begin{itemize}
\item Euclidean GCD
\item Binary GCD
\item 8bit Block Binary GCD
\item 8-bit Lehmer GCD
\item Left-to-Right Binary GCD
\end{itemize}

\subsection{Double-Base representations}
\subsubsection{Exponentiating by the Power Primorial}
\begin{itemize}
\item NAF-R2L
\item $\pm$ 2,3 Chains R2L $\pmod{3}$
\item $\pm$ 2,3 Chains R2L $\pmod{36=2^2 3^2}$
\item $\pm$ 2,3 Chains L2R (Ostrowski based)
\item Optimal strictly chained +2,3 partitions (incorporates cost).
\item $\pm$ 2,3 Representation using best 2,3 approximation of $N$, i.e. $\min \{|N-2^a3^b|\}$.
\item $\pm$ 2,3 Representation using $\min \left\{ \left| \frac{N \pm 1}{2^c3^d} \right| \right\}$.
\item $\pm$ 2,3 Representation using $\min \left\{ \left| \frac{N \pm 2^a3^b}{2^c3^d} \right| \right\}$.
\item Iterating on the \emph{best} 16 results for each of the greedy $\pm$ 2,3 representation algorithms: $\min \{|N-2^a3^b|\}$, $\min \left\{ \left| \frac{N \pm 1}{2^c3^d} \right| \right\}$, and $\min \left\{ \left| \frac{N \pm 2^a3^b}{2^c3^d} \right| \right\}$.
\item Precomputing \emph{optimal} $\pm$ 2,3 representations for 16bit numbers
	\begin{itemize}
	\item algorithm, incoporating costs, cache efficiency
	\item 16-bit blocking of primorial
	\item comparison of precomputed representation with 
	\end{itemize}
\end{itemize}

\subsubsection{Exponentiation by the list of Prime Powers}
\begin{itemize}
\item The primorial bounds used generate prime powers all $\le$ 16-bit.  As such, powering by precomputed representations is fastest when exponentiating by a list of prime powers.
\item Possible advantage is that we can stop exponentiating once the identity element is reached.
\item Disadvantaged is that we must first compute $a' = a^{2^{e_2}}$, then power $a'$ by each of the prime powers in the list, and then recompute $a'' = a^{{p_i}^{e_i}}$ for each prime power ${p_i}^{e_i}$ used, and square $a''$ until an ambiguous form is found.
\end{itemize}



\subsection{Search for best parameters for SuperSPAR}
\begin{itemize}
\item Best primorial bound for $n$-bit integers.
\item Best number of primorial steps relative to size of primorial.
\item Sequential prime forms vs Random prime forms.
\item Ratio of prime forms to multipliers.
\item Best multipliers to use.
\item Exponentiating by Power Primorial vs list of Prime Powers.
\end{itemize}



\bigbreak
\section{Empirical Results}

\subsection{Extended GCD}
\subsection{Power Primorial Exponentiation}
\subsection{Parameters for SuperSPAR}

\subsection{Comparison of factoring speeds with other algorithms}
\begin{itemize}
\item SQUFOF from Pari/GP
\item GNU MSieve
\item GMP-ECM
\item Maple
\item Sage (I believe Sage just uses Pari/GP)
\item Jerome Milan (TIFA) (If available)
\end{itemize}


\bigbreak
\section{Further Research}
\begin{itemize}
\item Parallelization using multipliers and a GPU
\end{itemize}


\end{document}

