\documentclass[11pt, letterpaper]{article}

\usepackage{algorithmic}
\usepackage{amsfonts}
\usepackage{amssymb}
\usepackage{amsmath}
\usepackage{amsthm}
\usepackage{fullpage}
\usepackage{comment}

\newtheorem*{thm}{Theorem}
\newtheorem*{lem}{Lemma}
\newtheorem*{cor}{Corollary}
\theoremstyle{definition}
\newtheorem*{defn}{Definition}


\parindent 0ex

\newcommand{\CC}{\mathbb{C}}
\newcommand{\NN}{\mathbb{N}}
\newcommand{\RR}{\mathbb{R}}
\newcommand{\RRgtz}{\mathbb{R}_{>0}}
\newcommand{\ZZ}{\mathbb{Z}}
\newcommand{\ZZgtz}{\mathbb{Z}_{>0}}
\newcommand{\ZZgez}{\mathbb{Z}_{\ge 0}}
\newcommand{\QQ}{\mathbb{Q}}
\newcommand{\QQgtz}{\mathbb{Q}_{>0}}
\newcommand{\QQgez}{\mathbb{Q}_{\ge 0}}
\newcommand{\NP}{\textrm{NP}}
\newcommand{\matrixot}[2]{\left( \begin{array}{r} #1 \\ #2 \end{array} \right)}
\newcommand{\matrixtt}[4]{\left( \begin{array}{rr} #1 & #2 \\ #3 & #4 \end{array} \right)}
\newcommand{\ntoinfty}{\lim_{n \rightarrow \infty}}
\newcommand{\floor}[1]{\left\lfloor #1 \right\rfloor}
\newcommand{\ceil}[1]{\left\lceil #1 \right\rceil}


\begin{document}

% MOTIVATION
\section{Motivation}
\begin{itemize}
\item Motivate faster ideal exponentiation in imaginary quadratic fields
\item Exponentiation with fixed exponents (when precomputation is okay).
\item Faster class group computations
\item Application in cryptography
\item Example of fast ideal exponentiation in SuperSPAR
\end{itemize}


% IDEAL EXPONENTIATION
\section{Ideal Exponentiation}
\begin{itemize}
\item Compose, Square, Cube

\item Quadratic Forms
	\begin{itemize}
	\item Identity
	\item Inverse
	\item Prime Forms
	\item Ambiguous Forms
	\end{itemize}
	

\item Double-Base Number Systems
	\begin{itemize}
	\item When/why it is useful?
	\item Chains vs Representations
	\item Encoding of chains/representations
	\item Sub-linearity in number of terms
	\end{itemize}
	
	
\end{itemize}

% SUPERSPAR
\bigbreak
\section{SuperSPAR}
\begin{itemize}
\item As an example of ideal exponentiation
\item SPAR Algorithm
	\begin{itemize}
		Complexity
	\end{itemize}
\item Primorial Steps Algorithm
\item SuperSPAR Algorithm
	\begin{itemize}
		Complexity
	\end{itemize}
\end{itemize}


% EXPERIMENTS
\bigbreak
\section{Experiments}

\begin{itemize}
\item Motivation: What is the best way that we know to do ideal exponentiation?  What is the best way to apply these results to SuperSPAR?

\item Extended GCD
	\begin{itemize}
	\item Euclidean GCD (partial)
	\item Lehmer GCD (partial)
	\item Binary GCD (non-partial)
	\item 8-bit Block Binary GCD (non-partial)
	\item Left-to-Right Binary GCD (partial)
	\item 32bit, 64bit, 128bit
	\end{itemize}
	
	
\item Quadratic Form Arithmetic
	\begin{itemize}
	\item 64bit, 128bit, GMP-MPZ
	\item When is cubing better than compose+square?
	\item Average time for operations
	\end{itemize}	
	
\item Exponentiation
	\begin{itemize}
	\item NAF-R2L
	\item $\pm$ 2,3 Chains R2L $\pmod{3}$
		\begin{itemize}
		\item Fast $\pmod 3$.
		\item Fast divide by 3.
		\end{itemize}
	\item $\pm$ 2,3 Chains R2L $\pmod{36=2^2 3^2}$
		\begin{itemize}
		\item Fast $\pmod{36}$.
		\end{itemize}
	\item $\pm$ 2,3 Chains L2R (Ostrowski based)
	\item Optimal +2,3 strictly chained partitions (incorporates cost).
	\item $\pm$ 2,3 Representation using best 2,3 approximation of $N$, i.e. $\min \{|N-2^a3^b|\}$.
	\item $\pm$ 2,3 Representation using $\min \left\{ \left| \frac{N \pm 1}{2^c3^d} \right| \right\}$.
	\item $\pm$ 2,3 Representation using $\min \left\{ \left| \frac{N \pm 2^a3^b}{2^c3^d} \right| \right\}$.
	\item $\pm$ 2,3 Representation using $\min \left\{ \left| \frac{N \pm 2^a \pm 3^b}{2^c3^d} \right| \right\}$.
	\item Iterating on the \emph{best} 16 (or $x$) results for each of the greedy $\pm$ 2,3 representation algorithms: $\min \{|N-2^a3^b|\}$, $\min \left\{ \left| \frac{N \pm 1}{2^c3^d} \right| \right\}$, $\min \left\{ \left| \frac{N \pm 2^a3^b}{2^c3^d} \right| \right\}$, and $\min \left\{ \left| \frac{N \pm 2^a \pm 3^b}{2^c3^d} \right| \right\}$.
	\item Precomputing \emph{optimal} $\pm$ 2,3 representations for 16bit numbers
		\begin{itemize}
		\item algorithm, incoporating costs, cache efficiency
		\item 16-bit blocking of exponent
		\item comparison of precomputed representation with greedy approaches
		\end{itemize}
	
	\item By fixed primorial
	\item By list of prime powers
	\item By general integers (maybe some experiments)
	\end{itemize}

\item SuperSPAR
	\begin{itemize}

	\item Coprime Finding
		\begin{itemize}
		\item Wheeling
		\item Sieving
		\item Distance Table
		\end{itemize}

	\item Exponentiating 
		\begin{itemize}
		\item Primorial vs Prime Powers
		\item Best primorial bound for $n$-bit integers.
		\end{itemize}
		
	\item Time spent on search vs powering.
	\item Sequential prime forms vs Random prime forms.
	\item Ratio of prime forms to multipliers.
	\item Best multipliers to use.
	
	\item Compare with other algorithms
		\begin{itemize}
		\item SQUFOF from Pari/GP
		\item GNU MSieve
		\item GMP-ECM
		\item Maple
		\item Sage (I believe Sage just uses Pari/GP)
		\item Jerome Milan (TIFA) (If available)
		\end{itemize}

	\end{itemize}

\end{itemize}


% EMPIRICAL RESULTS
\bigbreak
\section{Empirical Results}

\begin{itemize}
\item Follows the order in the section on Experiments
\end{itemize}


% FURTHER RESEARCH
\bigbreak
\section{Further Research}
\begin{itemize}
\item Other types of Ideal Arithmetic?
\item Function Fields?
	\begin{itemize}
	\item Hyperelliptic Curves?
	\end{itemize}
\item DBNS with other coprime bases?
\item NS with three,four,etc bases?
\item Further application to SuperSPAR?
	\begin{itemize}
	\item Parallelization using multipliers and a GPU?
	\end{itemize}
\end{itemize}


\end{document}

