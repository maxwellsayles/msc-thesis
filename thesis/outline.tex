\documentclass[11pt, letterpaper]{article}

\usepackage{algorithmic}
\usepackage{amsfonts}
\usepackage{amssymb}
\usepackage{amsmath}
\usepackage{amsthm}
\usepackage{fullpage}
\usepackage{comment}

\newtheorem*{thm}{Theorem}
\newtheorem*{lem}{Lemma}
\newtheorem*{cor}{Corollary}
\theoremstyle{definition}
\newtheorem*{defn}{Definition}


\parindent 0ex

\newcommand{\CC}{\mathbb{C}}
\newcommand{\NN}{\mathbb{N}}
\newcommand{\RR}{\mathbb{R}}
\newcommand{\RRgtz}{\mathbb{R}_{>0}}
\newcommand{\ZZ}{\mathbb{Z}}
\newcommand{\ZZgtz}{\mathbb{Z}_{>0}}
\newcommand{\ZZgez}{\mathbb{Z}_{\ge 0}}
\newcommand{\QQ}{\mathbb{Q}}
\newcommand{\QQgtz}{\mathbb{Q}_{>0}}
\newcommand{\QQgez}{\mathbb{Q}_{\ge 0}}
\newcommand{\NP}{\textrm{NP}}
\newcommand{\matrixot}[2]{\left( \begin{array}{r} #1 \\ #2 \end{array} \right)}
\newcommand{\matrixtt}[4]{\left( \begin{array}{rr} #1 & #2 \\ #3 & #4 \end{array} \right)}
\newcommand{\ntoinfty}{\lim_{n \rightarrow \infty}}
\newcommand{\floor}[1]{\left\lfloor #1 \right\rfloor}
\newcommand{\ceil}[1]{\left\lceil #1 \right\rceil}


\begin{document}

% MOTIVATION
\section{Motivation}
\begin{itemize}
\item Faster ideal exponentiation on imaginary quadratic fields
\item Exponentiation with fixed exponents (when precomputation is okay).
\item Faster class group computations
\item Applications in cryptography
\item Example of fast ideal exponentiation in SuperSPAR
\item Summary of Contributations
\item Overview of Thesis
\end{itemize}


% IDEAL ARITHMETIC
\bigbreak
\section{Ideal Arithmetic}
\begin{itemize}
\item Imaginary Quadratic Fields
\item Maximal Order of Algebraic Integers
\item Ideals
\item Definition of Class Group and Group Law
\item Identity, Inverse, Prime Ideal, Order
\item Compose, Square, Cube
\item Computer Representation
	\begin{itemize}
	\item Explicit Algorithms
	\end{itemize}
\end{itemize}

% EXPONENTIATION
\bigbreak
\section{Exponentiation}
\begin{itemize}
\item Binary, NAF-R2L
\item Double-Base Number Systems
	\begin{itemize}
	\item When/why it is useful?
	\begin{itemize}
		\item Sub-linearity in number of terms
	\end{itemize}
	\item Chains vs Representations
	\item Encoding of chains/representations
	\end{itemize}
\item Methods for computing 2,3 Chains/Representations in the literature
	\begin{itemize}
	\item R2L Chains
	\item L2R Chains (Ostrowski approximation)
	\item Yao's method for representations
	\item Pruned Trees
	\end{itemize}
\end{itemize}


% SUPERSPAR
\bigbreak
\section{SuperSPAR}
\begin{itemize}
\item As an example of ideal exponentiation
\item Connection of Ambiguous Ideals to the Factorization of the Discriminant
\item SPAR Algorithm
	\begin{itemize}
		\item Complexity
	\end{itemize}
\item Primorial Steps Algorithm
	\begin{itemize}
	\item How is the primorial bound chosen?  Bound on primes? Bound on prime power?
	\item How is this done in the original SPAR?
	\item How is this done for generic groups?  Can we do better for the class group of imaginary quadratic number fields?
	\end{itemize}
\item SuperSPAR Algorithm
	\begin{itemize}
		\item Complexity
		\item Analysis for generic groups
		\item Analysis for Class group of imaginary quadratic number fields
		\item Reanlysis of original SPAR paper
	\end{itemize}
\end{itemize}


% IDEAL EXPONENTIATION EXPERIMENTS
\bigbreak
\section{Ideal Exponentiation Experiments}

\begin{itemize}
\item Motivation: What is the best way that we know to do ideal exponentiation?  

\item Extended GCD
	\begin{itemize}
	\item Euclidean GCD (w/ partial)
	\item Lehmer GCD (w/ partial)
	\item Binary GCD (w/o partial)
	\item 8-bit Windowed Binary GCD (w/o partial)
	\item Left-to-Right Binary GCD (w/ partial)
		\begin{itemize}
		\item Compute $\min_k \{ |u-2^k| \}$ (literature)
		\item Compute $|u-2^k|$ where $k=\floor{log_2 u}$ (our method)
		\end{itemize}
	\item 32bit, 64bit, 128bit Intel assembler
	\end{itemize}
	
	
\item Ideal Arithmetic
	\begin{itemize}
	\item 64bit, 128bit, GMP-MPZ
	\item When is cubing better than compose+square?
	\item Average time for operations
	\end{itemize}	
	
\item Exponentiation
	\begin{itemize}
	\item NAF-R2L
	\item $\pm$ 2,3 Chains R2L $\pmod{3}$
		\begin{itemize}
		\item Fast $\pmod 3$.
		\item Fast divide by 3.
		\end{itemize}
	\item $\pm$ 2,3 Chains R2L $\pmod{36=2^2 3^2}$
		\begin{itemize}
		\item Fast $\pmod{36}$.
		\end{itemize}
	\item $\pm$ 2,3 Chains L2R (Ostrowski based)
	\item Optimal +2,3 strictly chained partitions (incorporates cost).
	\item $\pm$ 2,3 Representation using best 2,3 approximation of $N$, i.e. $\min \{|N-2^a3^b|\}$.
	\item $\pm$ 2,3 Representation using $\min \left\{ \left| \frac{N \pm 1}{2^c3^d} \right| \right\}$.
	\item $\pm$ 2,3 Representation using $\min \left\{ \left| \frac{N \pm 2^a3^b}{2^c3^d} \right| \right\}$.
	\item $\pm$ 2,3 Representation using $\min \left\{ \left| \frac{N \pm 2^a \pm 3^b}{2^c3^d} \right| \right\}$.
	\item Iterating on the \emph{best} 16 (or $x$) results for each of the greedy $\pm$ 2,3 representation algorithms: $\min \{|N-2^a3^b|\}$, $\min \left\{ \left| \frac{N \pm 1}{2^c3^d} \right| \right\}$, $\min \left\{ \left| \frac{N \pm 2^a3^b}{2^c3^d} \right| \right\}$, and $\min \left\{ \left| \frac{N \pm 2^a \pm 3^b}{2^c3^d} \right| \right\}$.
	\item Precomputing \emph{optimal} $\pm$ 2,3 representations for 16bit numbers
		\begin{itemize}
		\item algorithm, incoporating costs, cache efficiency
		\item 16-bit blocking of exponent
		\item comparison of precomputed representation with greedy approaches
		\end{itemize}
	
	\item By fixed primorial
	\item By list of prime powers
	\item By general integers (maybe some experiments)
	\end{itemize}
\end{itemize}


% SUPERSPAR EXPERIMENTS
\bigbreak
\section{SuperSPAR Experiments}
\begin{itemize}
\item Motivation: How efficiently can we factor integers using SuperSPAR and the above improvements to ideal exponentiation?


\item Coprime Finding
	\begin{itemize}
	\item Wheeling
	\item Sieving
	\item Distance Table
	\end{itemize}

	\item Exponentiating 
	\begin{itemize}
	\item Bounds for primorial (Bound for prime vs prime power)
	\item Primorial vs Prime Powers
	\item Best primorial bound for $n$-bit integers.
	\end{itemize}
	
\item Time spent on search vs powering.
\item Sequential prime ideals vs Random prime ideals.
\item Ratio of prime ideals to multipliers.
\item Best multipliers to use.

\item Compare with other algorithms
	\begin{itemize}
	\item SQUFOF from Pari/GP
	\item GNU MSieve
	\item GMP-ECM
	\item Maple
	\item Sage (I believe Sage just uses Pari/GP)
	\item Jerome Milan (TIFA) (If available)
	\end{itemize}

\end{itemize}




% FURTHER RESEARCH
\bigbreak
\section{Further Research}
\begin{itemize}
\item Other types of Ideal Arithmetic?
	\begin{itemize}
	\item The Ideal Ring of Algebraic Integers of Real Quadratic Fields
	\end{itemize}
\item Function Fields?
	\begin{itemize}
	\item Hyperelliptic Curves?
	\end{itemize}
\item DBNS with other coprime bases?
\item Number Systems with three,four,etc. bases?
\item Further application to SuperSPAR?
	\begin{itemize}
	\item Parallelization using multipliers and a GPU?
	\end{itemize}
\end{itemize}


\end{document}

